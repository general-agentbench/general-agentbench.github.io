\documentclass{article}



\usepackage{arxiv}
\usepackage[utf8]{inputenc} % allow utf-8 input
\usepackage[T1]{fontenc}    % use 8-bit T1 fonts
\usepackage{hyperref}       % hyperlinks
\usepackage{url}            % simple URL typesetting
\usepackage{booktabs}       % professional-quality tables
\usepackage{amsfonts}       % blackboard math symbols
\usepackage{nicefrac}       % compact symbols for 1/2, etc.
\usepackage{microtype}      % microtypography
\usepackage{lipsum}		% Can be removed after putting your text content
\usepackage{graphicx}
\usepackage{natbib}
\usepackage{doi}
\usepackage{tabularx}
\usepackage{makecell}
\usepackage[table]{xcolor}
\usepackage{multirow}
\usepackage{amsmath}
\usepackage{subcaption}
\usepackage[many,most]{tcolorbox}
\usepackage{siunitx} % Critical for aligning decimals

\newtcolorbox[auto counter, number within=section]{NewBox}[2]{%
  float*, width=\textwidth,
  colback=white, colframe=black,
  colbacktitle=white, coltitle=black,
  fonttitle=\bfseries,
  boxrule=1.0pt,
  leftupper=0.5em, rightupper=0.5em,
  title={#1},
  label={#2},
}



\newcommand{\haok}[1]{{\color{purple}\textbf{haok:} #1}}
\newcommand{\andy}[1]{{\color{magenta}\textbf{andy:} #1}}
\newcommand{\abhijay}[1]{{\color{teal}\textbf{abhijay:} #1}}
\newcommand{\pranav}[1]{{\color{orange}\textbf{pranav:} #1}}
\newcommand{\xiaochuan}[1]{{\color{red}\textbf{xiaochuan:} #1}}
\newcommand{\ryan}[1]{{\color{cyan}\textbf{ryan:} #1}}




% \title{Beyond Tokens: An Empirical study on LLMs’ long-horizon agent performance}
\title{Agentic LongBench: Evaluating Test-Time Scaling Behavior of Language Models in Agentic Tasks}

%\date{September 9, 1985}	% Here you can change the date presented in the paper title
%\date{} 					% Or removing it

% \author{ \href{https://orcid.org/0000-0000-0000-0000}{\includegraphics[scale=0.06]{orcid.pdf}\hspace{1mm}David S.~Hippocampus}
% % \thanks{Use footnote for providing furtherinformation about author (webpage, alternativeaddress)---\emph{not} for acknowledging funding agencies.}
%     \\
% 	Department of Computer Science\\
% 	Cranberry-Lemon University\\
% 	Pittsburgh, PA 15213 \\
% 	\texttt{hippo@cs.cranberry-lemon.edu} \\
% 	%% examples of more authors
% 	\And
% 	\href{https://orcid.org/0000-0000-0000-0000}{\includegraphics[scale=0.06]{orcid.pdf}\hspace{1mm}Elias D.~Striatum} \\
% 	Department of Electrical Engineering\\
% 	Mount-Sheikh University\\
% 	Santa Narimana, Levand \\
% 	\texttt{stariate@ee.mount-sheikh.edu} \\
% 	%% \AND
% 	%% Coauthor \\
% 	%% Affiliation \\
% 	%% Address \\
% 	%% \texttt{email} \\
% 	%% \And
% 	%% Coauthor \\
% 	%% Affiliation \\
% 	%% Address \\
% 	%% \texttt{email} \\
% 	%% \And
% 	%% Coauthor \\
% 	%% Affiliation \\
% 	%% Address \\
% 	%% \texttt{email} \\
% }

% Uncomment to remove the date
%\date{}

% Uncomment to override  the `A preprint' in the header
%\renewcommand{\headeright}{Technical Report}
%\renewcommand{\undertitle}{Technical Report}
\renewcommand{\shorttitle}{\textit{arXiv} Template}
\definecolor{midnightgreen}{rgb}{0.0, 0.29, 0.33}
\newcommand{\cx}[1]{\textcolor{midnightgreen}{\bf\small [#1 --cx]}}
%%% Add PDF metadata to help others organize their library
%%% Once the PDF is generated, you can check the metadata with
%%% $ pdfinfo template.pdf
% \hypersetup{
% pdftitle={A template for the arxiv style},
% pdfsubject={q-bio.NC, q-bio.QM},
% pdfauthor={David S.~Hippocampus, Elias D.~Striatum},
% pdfkeywords={First keyword, Second keyword, More},
% }

\begin{document}
\maketitle

\begin{abstract}
% and therefore require language models to possess robust multi-turn reasoning capabilities
Agentic tasks—such as codebase debugging and web information retrieval—naturally involve long contexts and represent a key bottleneck in evaluating and deploying LLM capabilities in real-world settings. While prior work has shown that test-time scaling can be effective for them, increasing inference-compute also causes interaction histories to grow: multiple rounds of model generation and environmental feedback become interleaved, forming complex, evolving contexts that are not adequately captured by existing synthetic, text-only long-context benchmarks. To address this gap, we introduce Agentic LongBench, evaluating frontier LLMs on diverse agentic tasks with long and easily extensible context. Additionally, we study two primary test-time scaling paradigms on our benchmark—parallel scaling and sequential scaling—to characterize model scaling behavior. By extending context lengths up to 256K tokens and sampling up to eight trajectories, we find that allocating compute to self-refinement yields limited performance gains, whereas increasing the number of samples leads to substantial improvements. Additionally, we provide two further analyses—attention analysis and agent design analysis—to demonstrate how our benchmark can be used to study design factors that influence models’ agentic capabilities.
\cx{switch to ICML template???}
% We further compare full attention, sparse attention, and linear attention mechanisms to analyze how attention types impact long-context agentic performance. Finally, we propose an all-in-one agentic system that removes task-specific prompts and allows the model to autonomously determine its strategy. Our findings offer new insights into the scaling behavior of LLMs on long-context agentic tasks and provide a foundation for evaluating and designing models capable of sustained, multi-turn interaction with complex environments.

% However, existing long-context benchmarks either over-simplify task settings—as in synthetic problems (e.g., Needle-in-a-Haystack) and purely textual evaluations (e.g., paragraph reranking)—or fail to reflect the interaction patterns of real-world agentic tasks, which are inherently multi-turn rather than single-turn.


% It covers four agentic categories and ensures each task involves sufficiently long interaction trajectories. 
\end{abstract}
% , including both dense and sparse attention architectures. 


% keywords can be removed
% \keywords{First keyword \and Second keyword \and More}


\begin{abstract}

True agentic ability requires solving long-horizon agentic tasks without reliance on domain-specific hints. However, existing benchmarks often suffer from domain leakage, where provided tool descriptions inadvertently introduce prior knowledge bias. To address this, we present Omni AgenticBench, a benchmark spanning four domains that explicitly removes such leakage to enable fair evaluation. Additionaly, we systematically investigate test-time scaling strategies, revealing distinct and often counterintuitive performance trends across models. Beyond standard pass@k metrics, we introduce a self-choice evaluation, where models identify the correct final answer from multiple generated candidates. Our results expose a substantial gap between the models' solution space and their self-cognition space. Finally, we analyze attention mechanisms of recent linear and sparse architectures under agentic settings. All tasks, code, and data are publicly available.

\end{abstract}


\section{Introduction}

% \cx{make this a benchmark paper? then the studies are insights we have with the benchmark. title is not reflecting this.}

Developing general-purpose agents powered by Large Language Models (LLMs) has emerged as a primary focus of current AI research\cite{yao2022react, liu2023agentbench, schick2023toolformer}. A defining characteristic of agentic tasks is their reliance on extensive context; predefined tool descriptions, complex user queries, model-generated reasoning, and environmental feedback collectively form intricate, long-context, multi-turn interaction histories\cite{qin2023toolllm,zhou2023webarena,yang2024sweagent}. Consequently, the performance of these agents is inextricably linked to the model’s ability to reason over both tool specifications and long-horizon information.

While agentic benchmarks are well-established for evaluating frontier models, their evaluation protocols often suffer from a critical shortcoming. To achieve true generality, future agents must solve problems without prior assumptions regarding the task content\cite{liang2025sweillusion} or category. However, existing benchmarks typically provide only domain-specific tool descriptions for a given task, resulting in a constrained and narrow evaluation environment. For instance, SWE-bench\cite{jimenez2023swebench} provides only bash commands tools in their policy prompts to facilitate software engineering tasks, while BrowserComp\cite{wei2025browsecomp} defines only a search tool to assist with web navigation without other tools' interference. We argue that these predefined toolsets unintentionally signal the task's domain to the agent—a phenomenon we term \textbf{``domain leakage''}—thereby simplifying the problem and failing to test the agent's ability to operate in a truly open-ended environment.

In this paper, we bridge this gap by introducing Omni AgenticBench, a benchmark designed to comprehensively evaluate agentic capabilities across diverse categories—including search, coding, reasoning, and tool-use—within a unified toolset framework. We began by auditing widely used agent benchmarks and evaluating frontier open-source models to ensure our task selection was both high-quality and representative of state-of-the-art challenges. To eliminate domain leakage, we consolidated tool definitions and descriptions from all domains into a single, unified library. We then replaced domain-specific policy prompts and constrained toolsets with this global library and a set of general instructions. This configuration allows us to decouple an agent's true reasoning ability from the performance gains provided by domain-specific hints. Our comprehensive evaluation of ten leading LLMs reveals a significant performance degradation for most agents when prior domain hints are removed. Interestingly, a subset of models maintains or even marginally improves performance, suggesting a higher degree of robust, generalizable agency. Furthermore, because our merged "Omni-toolset" accumulates to approximately 64K tokens, the benchmark inherently serves as a rigorous test of long-context utilization. By analyzing the correlation between Omni AgenticBench and traditional long-context benchmarks—such as Needle In A Haystack\cite{kamradt_needlehaystack_2023} and long-document QA\cite{bai2025longbenchv2}—we demonstrate that high performance on static long-context tasks does not necessarily transfer to the dynamic requirements of agentic workflows.

% We evaluate not only frontier large language models (LLMs) on Agentic LongBench to ground their long-context abilities in agentic scenarios, but also include recent innovations in attention design—such as Qwen3-Next and DeepSeek-v3.2—to reveal how modifications to the attention mechanism affect performance in stressful real-world cases. Although the adoption of sparse attention and linear-attention mixture architectures aims for higher efficiency and reduced memory bottlenecks, their performance still lags significantly behind established models like GPT-5 and Claude 4.5 Sonnet, highlighting a persistent performance gap for these attention variants. Furthermore, we evaluate these models on existing representative long-context benchmarks and compare their score correlations with our findings. The low correlation between the two suggests that single-turn long-context proficiency does not directly transfer to agentic settings. To provide a more robust assessment, we introduce the Consistency-Accuracy Index (CAI), which balances raw performance with stability. Our findings show that, with the notable exception of GPT-5, most models exhibit unstable ranking trends, struggling to maintain high mean accuracy alongside behavioral consistency.

% While the context window of backbone LLMs has scaled to millions of tokens, effectively eliciting and evaluating their long-context potential remains an open challenge. Existing benchmarks designed to assess long-context abilities suffer from two critical limitations: (1) Task Distribution Mismatch: Traditional benchmarks often rely on long-document Question Answering (QA) or summarization, simply scaling the input size without incorporating the core elements of agentic workflows—namely, dynamic environments, tool-use, and nuanced user intents. (2) Structural Mismatch: Current benchmarks are predominantly single-turn. This format fails to capture the complexity of agentic tasks, where the long context is composed of the model’s own historical actions, such as command executions, intermediate summaries, and self-reflections. These recursive dependencies are fundamentally absent in single-turn evaluation datasets, therefore failed to mimic real world agent tasks. Furthermore, current agent related benchmarks often suffer from scope constraints: they either lack tool integration, which artificially restricts the model's action space, or focus exclusively on niche domains such as machine learning engineering or gaming. Consequently, they fail to provide a comprehensive analysis of the relationship between task complexity and context length.

% In this paper, we address these gaps by introducing Agentic LongBench, a benchmark specifically curated to comprehensively evaluate agentic capabilities across diverse task categories, including search, coding, reasoning, and tool-use domains. We evaluate not only frontier large language models (LLMs) on Agentic LongBench to ground their long-context abilities in agentic scenarios, but also include recent innovations in attention design—such as Qwen3-Next and DeepSeek-v3.2—to reveal how modifications to the attention mechanism affect performance in stressful real-world cases. Although the adoption of sparse attention and linear-attention mixture architectures aims for higher efficiency and reduced memory bottlenecks, their performance still lags significantly behind established models like GPT-5 and Claude 4.5 Sonnet, highlighting a persistent performance gap for these attention variants. Furthermore, we evaluate these models on existing representative long-context benchmarks and compare their score correlations with our findings. The low correlation between the two suggests that single-turn long-context proficiency does not directly transfer to agentic settings. To provide a more robust assessment, we introduce the Consistency-Accuracy Index (CAI), which balances raw performance with stability. Our findings show that, with the notable exception of GPT-5, most models exhibit unstable ranking trends, struggling to maintain high mean accuracy alongside behavioral consistency.

The multi-turn interaction format of agentic tasks make it easy for agents to allocate more compuation and enable test-time scaling: by forcing more turns and encouraging agents to reflect, we expect models to utilize their reasoning abilities to revise answers, retrieve missing information from the context, or develop new reasoning paths. While test-time scaling trend has been well studied for pure reasoning tasks, the behavior of agent scaling remains under-explored. We provide a systematic study of two primary scaling methods in agentic settings: (1) parallel Scaling: independently sampling $K$ trajectories and calculating accuracy via the best@$K$ metric. (2) sequential scaling: monitoring the model's "closing intent" and manually intervening with an additional turn of encouragement to prompt further reflection and thought. Our results across five models demonstrate that while parallel scaling follows expected improvement trends, sequential scaling exhibits the opposite phenomenon: nearly all models suffer performance degradation even with four times the computation. In some cases, sequential scaling performance falls below the baseline evaluation. Further analysis reveals that current LLMs either struggle to identify their own errors buried within the context or fail to maintain their initial correct answers, eventually overturning them. This indicates that even frontier models struggle to reason effectively over raw, accumulated trajectory contexts.

While agents often achieve high best@$K$ scores, this metric merely indicates that a human can verify the task and that a correct solution exists within the agent’s solution space. It provides no guarantee that the agent can identify that correct generation. For parallel scaling to be effective, an agent must not only sample the correct answer but also "recognize" it. This capability is essential for agents to improve autonomously, particularly in non-verifiable tasks. To address this, we move beyond simple best@$K$ metrics to evaluate "self-selection"—a setting where the agent must evaluate its own candidates and select a final answer. We examine two selection methodologies: point-wise, where the agent evaluates generations individually, and pair-wise, where the agent compares two generations at a time using an iterative approach similar to bubble sort. Under this framework, a task is successful only if a correct generation is both produced and selected by the model. Our results reveal a significant gap between best@$K$ results and self-selection accuracy, highlighting a critical need to improve agent self-cognition to better align a model's judgment with its internal solution space.

% Consequently, agentic tasks inherently integrate long-context processing with reasoning. Their features to easily allocate additional computation makes them an ideal testbed for analyzing test-time scaling behavior. While test-time scaling has proven effective for pure reasoning tasks, its application to agentic workflows remains under-explored. We provide a systematic study of two primary scaling methods in agentic settings: (1) Parallel Scaling: Independently sampling $K$ trajectories and calculating accuracy via the Best@$K$ metric. (2) Sequential Scaling: Monitoring the model's "closing intent" and manually intervening with an additional turn of encouragement to prompt further reflection and thought. Our results across five models demonstrate that while parallel scaling follows expected improvement trends, sequential scaling exhibits the opposite phenomenon: nearly all models suffer performance degradation even with four times the computation. In some cases, sequential scaling performance falls below the baseline evaluation. Further analysis reveals that current LLMs either struggle to identify their own errors buried within the context or fail to maintain their initial correct answers, eventually overturning them. This indicates that even frontier models struggle to reason effectively over raw, accumulated trajectory contexts.

Finally, we demonstrate that Omni AgenticBench serves as a powerful diagnostic testbed through an attention analysis. Our study covers standard scaled dot-product attention as well as architectural innovations such as sparse and linear attention. We utilize a reasoning-behavior framework to identify critical reasoning spans within the input text, subsequently computing accumulated attention over these spans to extract the top-$k$ most influential tokens. By calculating the top-$k$ token overlap across layers (inter-layer) and heads (intra-layer), we quantify the functional diversity and redundancy inherent in different architectures. We further characterize the "effective receptive field" of these components by computing the average attention distance.  Finally, we provide case studies by mapping top-$k$ attention tokens back to the input text, revealing the specific contextual information that triggers reasoning behaviors. In summary, our contributions are:

% Finally, we demonstrate that Omni AgenticBench can serve as a powerful diagnostic testbed by providing an additional attention patterns analysis. We adopt a reasoning-behavior framework to identify key reasoning spans in the text first. Then, by computing accumulated attention over these spans and mapping top-$k$ scores back to tokens, we investigate which contextual information triggers reasoning. While some high-attention tokens correspond to relevant content (e.g., location tokens triggering geographical query modifications), many high-scoring tokens lack semantic meaning, reflecting classic issues such as attention sinks and local attention bias. In our context engineering analysis, we contrast task-specific tool availability with an "Omni-setting" (providing all available tools regardless of the task). This removes the prior bias of pre-filtered toolsets. Our results show a nearly universal performance drop in the Omni-setting, challenging the community to build agents that are truly environment-agnostic. In summary, our contributions are:



\begin{itemize}
    \item \textbf{Omni AgenticBench Framework:} We introduce Omni AgenticBench, a novel evaluation framework designed to eliminate the prevalent issue of \textbf{domain leakage}. By decoupling domain-specific hints from task execution, it provides a rigorous and unbiased assessment of an agent’s ability to reason over long-horizon contexts across diverse domains.
    \item \textbf{Characterization of Test-Time Scaling:} We present a systematic study of test-time scaling behaviors in agentic settings. Our analysis identifies a critical bottleneck in model self-cognition: while parallel scaling expands the solution space, models frequently fail to identify the correct answer among their samples. We also reveal that sequential scaling often leads to performance degradation in complex agentic workflows.
    \item \textbf{Mechanistic Analysis across Attention Architectures:} We utilize Omni AgenticBench as a diagnostic testbed to uncover distinct attention patterns across various architectures, including linear and sparse attention mechanisms. Through our reasoning-behavior framework, we quantify the functional diversity and "effective receptive fields" of different layers and heads, mapping specific contextual triggers to reasoning behaviors.
\end{itemize}

% First, mechanistic analysis reveals a weak correlation between high-attention tokens and semantic reasoning, often overshadowed by attention sinks and local bias. Second, by introducing the "Omni-setting," we demonstrate a universal performance drop when agents are not provided with pre-filtered toolsets, highlighting the need for more environment-agnostic agent architectures.

\section{General AgentBench}

In this section, we introduce the construction of General AgentBench (Section~\ref{2.1}) and the unified evaluation framework (Section~\ref{2.2} ). Detailed prompt templates, tool specifications, and the unified policy are deferred to the Appendix~\ref{appendix:prompt}.

% \subsection{Domains and Sources}\label{2.1}
% Omni AgenticBench covers four major task categories: \textbf{Coding}, \textbf{Search}, \textbf{Tool-use}, and \textbf{Reason}. We select data sources that are widely used as reference benchmarks when new proprietary or open-weight models are released to ensure the quality. Table \ref{tab:composition} reports the statistics of our benchmark. 

% \cx{need to discuss what the role of data sources are for us}\xiaochuan{check}

% \cx{always use section intro to keep reader organized, you have space else where}

\subsection{Domains and Sources}\label{2.1}

Our benchmark spans four task domains: \textbf{Coding}, \textbf{Search}, \textbf{Tool-use}, and \textbf{Reason}. These domains reflect common real-world applications such as software engineering, information seeking, service workflows, and analytical reasoning, positioning General AgentBench as an initial step toward evaluating general-purpose agents in open-ended and unified settings. Table~\ref{tab:composition} summarizes the benchmark composition.

\begin{table}[h]
\centering
\small
\caption{Composition of General AgentBench}
\label{tab:composition}
\resizebox{0.9\linewidth}{!}{%
\begin{tabular}{llrr}
\toprule
\textbf{Domain} & \textbf{Dataset} & \textbf{Original} & \textbf{Sampled} \\
\midrule
\multirow{2}{*}{Search}
  & BrowseComp   & 1266 & 124\\
  & WebVoyager   & 643  & 65 \\
\midrule
\multirow{2}{*}{Coding}
  & SWE-Bench Verified & 500 & 50 \\
  & Terminal-Bench     & 230 & 80 \\
\midrule
Reason
  & MathHay            & 602 & 75 \\
\midrule
\multirow{2}{*}{Tool-Calling}
  & Tau2-Bench         & 278 & 50 \\
  & MCP-Bench          & 104 & 52 \\
\bottomrule
\end{tabular}%
}
\end{table}

\paragraph{Coding}
We include tasks from SWE-Bench Verified \cite{openai_swebench_verified_2024} and Terminal Bench, which evaluate an agent’s ability to analyze production-level software issues, reason over long instructions, and iteratively interact with execution environments to reach a correct final state.

\paragraph{Search}
The search domain includes tasks from BrowseComp \cite{wei2025browsecomp} and WebVoyager \cite{he2024webvoyager}. These benchmarks assess an agent’s ability to identify missing information, decide when additional search steps are needed, and navigate long, evolving web contexts, going beyond static retrieval or single-turn question answering.

\paragraph{Tool-use}
For tool-use, we adopt Tau2-Bench \cite{barres2025tau2} and MCP-Bench \cite{wang2025mcpbench}, both of which provide rich tool suites requiring models to select, invoke, and coordinate multiple tools. These tasks emphasize structured tool calling and multi-step planning in realistic service and workflow scenarios.

\paragraph{Reason}
For long-context reasoning, we use MathHay \cite{wang2024mathhay}, which constructs queries by embedding relevant mathematical documents into noisy long-context haystacks. This benchmark isolates sustained reasoning over long inputs without relying on external tool execution, complementing the other domains.

% \cx{the discussion of these four existing categoties and benchmarks can be much shorter, more details can be moved to appendix, just general introduction are needed (and some highlight of why each task is good). Thus we have more space for our stuff.}\xiaochuan{check}

\begin{figure}[h]
    \centering
    \includegraphics[width=0.93\linewidth]{figs/bench_construction.pdf}
    \caption{\textbf{Illustration of how General AgentBench covers a wide range of task categories while providing a unified interface to simulate real-world user interactions.} The green region indicates the specific task currently being handled by the agent (e.g., a search task). Orange boxes denote other clients and servers that remain active and responsive but are not directly involved in the current interaction. Red indicates that other domain-specific data are excluded.}
    \label{fig:omni-setting}
\end{figure}

\begin{table*}[t]
\centering
\small
\caption{Main results on \textbf{General AgentBench}. Benchmarks are grouped by domain. \textbf{Avg.} denotes the mean score across all available benchmarks for each model. Bold indicates the best score.}
\label{tab:omni_only_domain_avg}
\setlength{\tabcolsep}{3.5pt}
\begin{tabular}{@{} l rr rr r rr r @{}}
\toprule
\multirow{2}{*}{\textbf{Models}} &
\multicolumn{2}{c}{\textbf{Search}} &
\multicolumn{2}{c}{\textbf{Code}} &
\textbf{Reason} &
\multicolumn{2}{c}{\textbf{Tool-use}} &
\multirow{2}{*}{\textbf{Avg.}} \\
\cmidrule(lr){2-3} \cmidrule(lr){4-5} \cmidrule(lr){6-6} \cmidrule(lr){7-8}
& \textbf{BrowseComp} & \textbf{WebVoyager}
& \textbf{SWE-Bench} & \textbf{Terminal-Bench}
& \textbf{MathHay}
& \textbf{Tau2-Bench} & \textbf{MCP-Bench}
& \\
\midrule

\textit{Open-Source} \\

GPT-OSS-120B
& 4.0 & 27.7 & 12.0 & 6.3 & 38.7 & 26.0 & 63.3 & 25.4 \\

Qwen3-235B-A22B
& 8.9 & 30.8 & 20.4 & 23.8 & 32.0 & 38.3 & 66.1 & 31.5\\

Qwen3-Next
& 10.5 & 35.4 & 18.0 & 8.8 & 42.0 & 48.9 & 64.6 & 32.6 \\

DeepSeek-V3.2
& 19.4 & 46.2 & 31.8 & 22.2  & 33.3 &  \bfseries
 54.0 &  66.0 & 39.0 \\

DeepSeek-R1
& 9.7 & 43.1 & 14.0 & 8.8 & 46.7 & 17.1 & 62.2 & 28.8 \\

\hline
% \addlinespace[0.5em]
\textit{Proprietary} \\

Gemini 2.5-Flash
& 6.5 & 32.3 & 14.0 & 20.0 & 36.0 & 38.3 & 66.6 & 30.5 \\

Gemini 2.5-Pro
& 8.9 & 46.2 & 26.0 & 27.5 & 24.0 & 46.0 & 67.2 & 35.1 \\

Claude Haiku 4.5
& 17.7 & 47.7 & \bfseries
 56.0 & 25.0 & 34.7 & 44.0 & 69.0 & 42.0 \\

Claude Sonnet 4.5
& 23.1 & 56.9 & 54.0 & \bfseries
 45.0 & 36.0 & 48.0 & \bfseries
 72.9 & \bfseries 48.0 \\

GPT-5
& \bfseries
 27.4 & \bfseries
 61.5 & 36.0 & 41.3 & \bfseries
 64.0 & 32.0 & 59.1 & 45.9 \\

\bottomrule
\end{tabular}
\end{table*}


\subsection{Unified Realistic Evaluation Framework}\label{2.2}

% These choices reflect three fundamental properties of real-world agent usage: cross-domain task diversity, comprehensive skill requirements, and dynamically evolving multi-turn interactions.

To support realistic evaluation of general LLM agents, we design a unified framework that exposes all tasks and tools through a shared interaction interface. These choices reflect three fundamental properties of real-world agent usage: cross-domain task diversity, comprehensive skill requirements, and dynamically evolving multi-turn interactions. An overview of the framework is illustrated in Figure~\ref{fig:omni-setting}.

\paragraph{Unified tool interface.}
In practical deployments, agents must select appropriate tools from a large pool without prior knowledge of task domains. To reflect this setting, we adopt the Model Context Protocol (MCP) \cite{mcp} as the backbone of our framework. Each benchmark environment is instantiated as an MCP server, while all servers are centrally managed by a unified Host. The Host maintains a global tool registry that records all available tools and their corresponding server mappings, presenting the agent with a single, unified tool space across all domains.

\paragraph{Centralized interaction abstraction.}
The Host serves as the sole interaction interface for the general agent, abstracting away individual benchmark implementations. When the agent invokes a tool, the Host resolves the call via the tool registry and routes the request to the appropriate server for execution. 

% This abstraction prevents the agent from directly accessing domain-specific environments and eliminates implicit domain cues, encouraging agents to infer task structure and tool relevance purely from interaction context.

\paragraph{Evolving interaction context.}
Because all tools and benchmark environments are exposed simultaneously, the unified tool descriptions alone can span tens of thousands of tokens. When combined with user queries and accumulated multi-turn interaction histories, the resulting context naturally grows into the long-context regime. In this setting, agents must reason over heterogeneous information sources, including task instructions, tool documentation, execution feedback, and their own prior decisions. This distinguishes agentic interaction from many existing long-context benchmarks that focus on static, single-turn question answering or summarization with short outputs. We provide further long-context study in Appendix ~\ref{appendix:long_compare}

\paragraph{Execution process.}
For each evaluation instance, the framework provides it to the agent together with the unified policy and toolset as the context. All benchmark servers (e.g., Docker-based environments in the coding domain) are instantiated simultaneously and remain idle while awaiting requests from the agent. When the agent issues a tool call, the Host routes the request to the corresponding server, executes the tool, and returns the result in a unified response format. Tool execution is decoupled from task type: even if a task is search-oriented, code-related tool calls can still be executed by the environment, returning valid outputs despite having no direct relevance to the final solution. This design intentionally exposes the agent to a realistic setting where incorrect or irrelevant tool usage remains possible.

The agent interacts with the framework over multiple turns until producing a final answer. During interaction, we monitor execution signals (e.g., terminal outputs) and regulate the interaction budget, enabling additional computation or extended reasoning when applicable (Section~\ref{4.1}). The final answer is then forwarded to the corresponding benchmark server for correctness evaluation.

% Together, these properties differentiate Omni AgenticBench from prior evaluation settings that isolate individual tasks, restrict tool access, or assume static interaction paradigms. As a result, performance measured under our unified framework better reflects agent behavior in realistic, open-ended user interactions. We further analyze the implications of these properties empirically in Section~\ref{3.3}. 


% \cx{we need to spend way more space in 2.2 as this is our core method, rather than spending too much space in 2.1 about preliminary.}\xiaochuan{check}

% \cx{this subsection focuses too much on the differences with one benchmark, in fact, give the story line, long-context benchmarks may not even be the most relevant related work... single agent tasks are... Instead of focusing too much on differences with one line of work, use this space to state the properties, advnatages and benefits of our work, and in each benefits we can make a contrast with previous tasks. Also maybe we should merge 2.3 and 2.2 as 2.3 is a reflection of 2.2}\xiaochuan{check}

\begin{figure*}[t]
    \centering
    \includegraphics[width=0.99\linewidth]{figs/omni_agenticbench_main.pdf}
    \caption{Relative performance change across domains from the Baseline ($B$) specialized agent setting to the general agent ($G$) setting with unified context and tools. Negative values indicate performance degradation under the General AgentBench.}
    \label{fig:domain_degradation}
\end{figure*}

\subsection{Experimental details}
% \cx{this is kind of experiemntal detail or benchmark constrution and more proper in sec 2?}\xiaochuan{check}
Our evaluation covers a total of ten frontier language models. Among open-source models, we include several high-performing systems such as Qwen3-235B \cite{yang2025qwen3} and DeepSeek-R1 \cite{guo2025deepseek}, as well as more recent models with novel attention mechanisms, including Qwen3-Next \cite{qwen3next} and DeepSeek-v3.2 \cite{liu2025deepseekv3.2}. For proprietary models, we consider both efficiency-oriented variants (e.g., Gemini 2.5 Flash \cite{comanici2025gemini}) and models optimized for complex reasoning (e.g., GPT-5 \cite{openai_gpt5} and Claude Sonnet 4.5 \cite{anthropic2025claude45}). We access these models via Amazon Bedrock \footnote{\url{https://aws.amazon.com/bedrock/pricing/}} and the Hugging Face Inference API .\footnote{\url{https://huggingface.co/docs/inference-providers/en/index}} For all evaluations, we fix the decoding temperature to 0.7 and ensure that each model’s native context length  exceeds the maximum context length required by the benchmark.

% This selection enables a systematic evaluation of emerging attention architectures in realistic agentic task environments, while allowing direct comparison with conventional attention mechanisms. 

% It is therefore necessary to explicitly distinguish long-context model performance in agentic settings from that in traditional long-context benchmarks.

% This makes Omni AgenticBench a rigorous testbed for agentic long-horizon capabilities. 
% Crucially, ``agentic long-context'' differs fundamentally from previous long-context benchmarks, which primarily evaluate single-turn, long-input comprehension. 



% In contrast, Omni AgenticBench grounds long-context evaluation in dynamic scenarios. Agentic tasks inherently generate expansive contexts through multi-turn interactions where the model must maintain a consistent trajectory. This shift moves beyond static assessment to measure a model’s ability to solve complex problems under realistic constraints. We provide further quantitative analysis in Section \ref{3.3}.



\section{Main Results}

In this section, we report overall performance on General AgentBench and compare it against evaluations conducted under prior domain-specific settings to quantify the gap between specialized and general-purpose LLM agents.

\subsection{Result analysis}

Table~\ref{tab:omni_only_domain_avg} summarizes the results across models and domains on General AgentBench. Claude Sonnet 4.5 achieves the strongest overall performance, driven primarily by its tool-use and coding capabilities, while GPT-5 attains the highest scores in the Search and Reason domains, reflecting its strengths in information retrieval and complex reasoning. Among open-source models, DeepSeek-V3.2 outperforms both Gemini variants, demonstrating the significant scaling potential of efficient, sparse-attention architectures. Across models, performance on BrowseComp remains consistently low, indicating that retrieving rare and precise information beyond in-domain training data is still a major bottleneck for current LLM agents.

\begin{figure}[t]
    \centering
    \includegraphics[width=0.95\linewidth]{figs/omni_agenticbench_summary.pdf}
\caption{Performance comparison between specialized-agent and general-agent settings.\textbf{Top}: Absolute performance .\textbf{Bottom}: Relative performance degradation under the general-agent setting.}
    \label{fig:mean_degradation}
\end{figure}

\begin{figure*}[t]
    \centering
    \includegraphics[width=0.92\linewidth]{figs/agentic_scaling_2x4.pdf}
    \caption{\textbf{Test-time scaling behaviors of general LLM agents.} Results are reported for five models across four domains on General AgentBench. \textbf{Top}: Parallel scaling expands the solution space through increased sampling. \textbf{Bottom}: Sequential scaling allocates additional computation via longer interaction histories, yet exhibiting unstable or diminishing returns.}
    \label{fig:tts}
\end{figure*}

We further examine how performance changes when models transition from specialized agents operating under domain-specific contexts to general agents acting within a unified environment with shared toolsets.  Figure~\ref{fig:mean_degradation} summarizes the mean degradation aggregated over all domains, while Figure~\ref{fig:domain_degradation} reports the relative performance change for each agent across domains. Most LLM agents experience substantial degradation in the general-agent setting, with average relative drops ranging from 10\% to 30\%. The magnitude of this degradation varies widely: for example, Gemini 2.5-Pro suffers a drop exceeding 60\% in the Reason domain, falling from top-tier performance in the baseline setting to near-average performance as a general agent. In contrast, Claude Sonnet 4.5 remains notably robust, with only a 0.2\% average degradation. Detailed overall results can be found in Appendix ~\ref{appendix:agentic_benchmark_details}.

\subsection{Cross-domain tool usage}\label{appendix:cross_domain_tool}

Interestingly, several models, including Qwen3-Next, Deepseek-R1, and Claude, exhibit \textbf{performance gains} in the Search domains under the general-agent setting. Trajectory-level analysis shows that these improvements arise from effective \textbf{cross-domain tool usage}, where agents repurpose tools beyond their originally intended domains to support reasoning and information retrieval. 
We take a closer look at these behaviors. Analysis of 189 search task traces from Claude Sonnet 4.5 reveals that $26$\% of tasks ($50/189$) utilized specialized domain tools beyond plain web search. The most frequently used specialized tools include Google Maps APIs ($78$ calls), Paper Search across arXiv, PubMed, and Google Scholar ($60$ calls), and Hugging Face model APIs ($36$ calls). We present a case study demonstrating how domain-specific tools outperform plain web search.

\begin{tcolorbox}[
  colback=white,
  colframe=gray!75,
  title=\textbf{Case Study: Retrieving the Latest HF Model},
  fonttitle=\bfseries\small,
  boxrule=0.6pt,
  left=3pt, right=3pt, top=4pt, bottom=4pt,
  sharp corners,
  breakable
]

{\small \textbf{Task.} Identify the most recent pre-trained language model on HuggingFace suitable for text classification.}

\vspace{5pt}
\begin{tcolorbox}[colback=red!3, colframe=red!15, left=2pt, right=2pt, top=2pt, bottom=2pt, boxrule=0.4pt]
\textbf{\scriptsize Plain Web Search Baseline (6 turns)} \\
\fontsize{6.5pt}{8pt}\selectfont
\texttt{T1: <search> latest Hugging Face pre-trained language\\ 
\phantom{T1: } model text classification 2024 </search>} \\
\texttt{T2: <search> Hugging Face text classification \\ 
\phantom{T2: } models sorted by recent trending 2024 </search>} \\
\texttt{T3: <search> Hugging Face newest text classification \\ 
\phantom{T3: } models January 2025 recently released </search>} \\
\texttt{...} \\
\texttt{T6: <answer> tabularisai/multilingual-sentiment- \\
\phantom{T6: } analysis (fine-tuned DistilBERT) </answer>}
\end{tcolorbox}

\vspace{3pt}

\begin{tcolorbox}[colback=blue!3, colframe=blue!15, left=2pt, right=2pt, top=2pt, bottom=2pt, boxrule=0.4pt]
\textbf{\scriptsize General Agent (3 turns)} \\
\fontsize{6.5pt}{8pt}\selectfont
\texttt{T1: Hugging\_Face\_\_search-models( \\
\phantom{T1: } \{"tags": "text-classification", "limit": $15$\}) \\
\phantom{T1: } $\rightarrow$ [\{"id": "FutureMa/Qwen3-4B-Evasion", \\
\phantom{T1: } "download": $15$\}, \\
\phantom{T1: } \{"id": "BAAI/bge-reranker-v2-m3", \\
\phantom{T1: } "download": $2807076$\}, ...]} \\
\texttt{T2: search\_\_web\_search("ModernBERT HF 2025") \\
\phantom{T2: } $\rightarrow$ "ModernBERT... $8192$-token context, \\
\phantom{T2: } $139$M/$395$M params, trained on $2$T tokens"} \\
\texttt{T3: Hugging\_Face\_\_get-model-info( \\
\phantom{T3: } \{"model\_id": "answerdotai/ModernBERT-base"\}) \\
\phantom{T3: } $\rightarrow$ Full model card and architecture details}
\end{tcolorbox}

\end{tcolorbox}

We observe that the plain search baseline iteratively refines web queries across 6 turns, ultimately finding \texttt{\small tabularisai/multilingual-sentiment-analysis} with only surface-level information (``\texttt{\small fine-tuned DistilBERT}''). In contrast, the General agent system's specialized \texttt{Hugging\_Face\_\_search-models} API directly queries the model hub with structured filters, returning download counts, tags, and model IDs. The subsequent \texttt{Hugging\_Face\_\_get-model-info} call retrieves comprehensive metadata including architecture specifications, training data scale, and official model cards---information unavailable through web search snippets. 

This behavior reflects an agent’s ability to dynamically select and compose tools under minimal domain priors, capturing a more realistic upper bound on general-agent capability and highlighting the importance of evaluation settings that approximate real-world tool availability. 



\input{src/4analysis}

\input{src/5multitask_system}

\bibliographystyle{unsrtnat}

\newpage
\input{src/6appendix}

\end{document}
