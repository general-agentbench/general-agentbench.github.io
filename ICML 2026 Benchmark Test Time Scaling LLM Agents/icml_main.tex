%%%%%%%% ICML 2026 EXAMPLE LATEX SUBMISSION FILE %%%%%%%%%%%%%%%%%

\documentclass{article}

% Recommended, but optional, packages for figures and better typesetting:
\usepackage{microtype}
\usepackage{graphicx}
\usepackage{subcaption}
\usepackage{booktabs} % for professional tables

% hyperref makes hyperlinks in the resulting PDF.
% If your build breaks (sometimes temporarily if a hyperlink spans a page)
% please comment out the following usepackage line and replace
% \usepackage{icml2026} with \usepackage[nohyperref]{icml2026} above.
\usepackage{hyperref}


% Attempt to make hyperref and algorithmic work together better:
\newcommand{\theHalgorithm}{\arabic{algorithm}}

% Use the following line for the initial blind version submitted for review:
% \usepackage{icml2026}

% For preprint, use
\usepackage[preprint]{icml2026}

% If accepted, instead use the following line for the camera-ready submission:
% \usepackage[accepted]{icml2026}

\usepackage{amsmath}
\usepackage{amssymb}
\usepackage{mathtools}
\usepackage{amsthm}


% if you use cleveref..
\usepackage[capitalize,noabbrev]{cleveref}

% self-added packages
\usepackage{tabularx}
\usepackage{makecell}
\usepackage{multirow}
\usepackage[many,most]{tcolorbox}
\usepackage{siunitx}
\usepackage[table]{xcolor}
\usepackage{svg} % Allows inclusion of .svg images
\usepackage{booktabs}
\usepackage{array}
\usepackage{arydshln}
\usepackage{xurl}
\usepackage{amsfonts}
\usepackage{algorithm}
\usepackage{algpseudocode}
\usepackage{setspace}

% \newtcolorbox[auto counter, number within=section]{NewBox}[2]{%
%   float*, width=\textwidth,
%   colback=white, colframe=black,
%   colbacktitle=white, coltitle=black,
%   fonttitle=\bfseries,
%   boxrule=1.0pt,
%   leftupper=0.5em, rightupper=0.5em,
%   title={#1},
%   label={#2},
% }


%%%%%%%%%%%%%%%%%%%%%%%%%%%%%%%%
% THEOREMS
%%%%%%%%%%%%%%%%%%%%%%%%%%%%%%%%
\theoremstyle{plain}
\newtheorem{theorem}{Theorem}[section]
\newtheorem{proposition}[theorem]{Proposition}
\newtheorem{lemma}[theorem]{Lemma}
\newtheorem{corollary}[theorem]{Corollary}
\theoremstyle{definition}
\newtheorem{definition}[theorem]{Definition}
\newtheorem{assumption}[theorem]{Assumption}
\theoremstyle{remark}
\newtheorem{remark}[theorem]{Remark}

% Todonotes is useful during development; simply uncomment the next line
%    and comment out the line below the next line to turn off comments
%\usepackage[disable,textsize=tiny]{todonotes}
\usepackage[textsize=tiny]{todonotes}

\definecolor{midnightgreen}{rgb}{0.0, 0.29, 0.33}
\newcommand{\cx}[1]{\textcolor{midnightgreen}{\bf\small [#1 --cx]}}
\newcommand{\xiaochuan}[1]{\textcolor{red}{\bf\small [#1 --xiaochuan]}}
\newcommand{\ryan}[1]{\textcolor{cyan}{\bf\small [#1 --ryan]}}
\newcommand{\haok}[1]{\textcolor{purple}{\bf\small [#1 --haok]}}
\newcommand{\andy}[1]{\textcolor{magenta}{\bf\small [#1 --andy]}}
\newcommand{\abhijay}[1]{\textcolor{teal}{\bf\small [#1 --abhijay]}}
\newcommand{\pranav}[1]{\textcolor{orange}{\bf\small [#1 --pranav]}}


% The \icmltitle you define below is probably too long as a header.
% Therefore, a short form for the running title is supplied here:
% \icmltitlerunning{Submission and Formatting Instructions for ICML 2026}

\begin{document}

\twocolumn[
  \icmltitle{Benchmark Test-Time Scaling of General LLM Agents}

  % It is OKAY to include author information, even for blind submissions: the
  % style file will automatically remove it for you unless you've provided
  % the [accepted] option to the icml2026 package.

  % List of affiliations: The first argument should be a (short) identifier you
  % will use later to specify author affiliations Academic affiliations
  % should list Department, University, City, Region, Country Industry
  % affiliations should list Company, City, Region, Country

  % You can specify symbols, otherwise they are numbered in order. Ideally, you
  % should not use this facility. Affiliations will be numbered in order of
  % appearance and this is the preferred way.
  \icmlsetsymbol{equal}{*}

  \begin{icmlauthorlist}
    \icmlauthor{Xiaochuan Li}{lti}
    \icmlauthor{Ryan Ming}{lti}
    \icmlauthor{Pranav Setlur}{lti}
    \icmlauthor{Abhijay Paladugu}{lti}
    \icmlauthor{Andy Tang}{lti}
    \icmlauthor{Hao Kang}{lti}
    \\
    \icmlauthor{Shuai Shao}{meta}
    \icmlauthor{Rong Jin}{meta}
    \icmlauthor{Chenyan Xiong}{lti}
  \end{icmlauthorlist}

  \icmlaffiliation{lti}{Language Technologies Institute, School of Computer Science, Carnegie Mellon University}
  \icmlaffiliation{meta}{Meta. All experiments, data collection, and processing activities were conducted by Carnegie Mellon University. Meta was involved solely in an advisory role.}
  % \icmlaffiliation{sch}{School of ZZZ, Institute of WWW, Location, Country}

  \icmlcorrespondingauthor{Xiaochuan Li}{xiaochu4@andrew.cmu.edu}
  % \icmlcorrespondingauthor{Firstname2 Lastname2}{first2.last2@www.uk}

  % You may provide any keywords that you find helpful for describing your
  % paper; these are used to populate the "keywords" metadata in the PDF but
  % will not be shown in the document
  \icmlkeywords{Machine Learning, ICML}

  \vskip 0.3in
]

% this must go after the closing bracket ] following \twocolumn[ ...

% This command actually creates the footnote in the first column listing the
% affiliations and the copyright notice. The command takes one argument, which
% is text to display at the start of the footnote. The \icmlEqualContribution
% command is standard text for equal contribution. Remove it (just {}) if you
% do not need this facility.

% Use ONE of the following lines. DO NOT remove the command.
% If you have no special notice, KEEP empty braces:
\printAffiliationsAndNotice{}  % no special notice (required even if empty)
% Or, if applicable, use the standard equal contribution text:
% \printAffiliationsAndNotice{\icmlEqualContribution}

\begin{abstract}

LLM agents are increasingly expected to function as general-purpose systems capable of resolving open-ended user requests. While existing benchmarks focus on domain-aware environments for developing specialized agents, evaluating general-purpose agents requires more realistic settings that challenge them to operate across multiple skills and tools within a unified environment. We introduce General AgentBench, a benchmark that provides such a unified framework for evaluating general LLM agents across search, coding, reasoning, and tool-use domains. Using General AgentBench, we systematically study test-time scaling behaviors under sequential scaling (iterative interaction) and parallel scaling (sampling multiple trajectories). Evaluation of ten leading LLM agents reveals a substantial performance degradation when moving from domain-specific evaluations to this general-agent setting. Moreover, we find that \textbf{neither scaling methodology yields effective performance improvements in practice}, due to two fundamental limitations: context ceiling in sequential scaling and verification gap in parallel scaling. Code is publicly available at \url{https://github.com/cxcscmu/General-AgentBench}.

% General agents must resolve real-world user requests that are complex in both content and task category. However, existing agentic evaluations primarily emphasize the former and are typically conducted in isolated environments, which naturally favor the assessment of specialized agents over general ones. To complement this evaluation paradigm, we introduce Omni AgenticBench, which provides a unified context and toolset for assessing general agents across diverse task categories within a single, coherent setting. Using Omni AgenticBench, we systematically study test-time scaling behaviors of general agents and show that, regardless of additional computation, current models struggle to iteratively refine and improve their responses. We further identify a substantial verification gap: although general agents can often sample correct solutions, they fail to reliably recognize them, preventing theoretical gains from test-time scaling from translating into effective performance improvements. Finally, we analyze the dominant failure modes that emerge when models are evaluated in the general-agent setting. All tasks, code, and data are publicly available.
% By evaluating frontier models through systematic test-time scaling, we demonstrate that current agents struggle to iteratively refine responses, regardless of computational budget. We further identify a substantial gap between the solution space agents can generate and their self-cognition space, thereby hindering their practical utility. Additionally, we analyze failed behaviors from the perspective of attention mechanisms, including a comparison between recently adopted sparse and linear attention. All tasks, code, and data are publicly available.

\end{abstract}


% Agentic tasks—such as codebase debugging and web information retrieval—naturally involve long contexts and represent a key bottleneck in evaluating and deploying LLM capabilities in real-world settings. While prior work has shown that test-time scaling can be effective for them, increasing inference-compute also causes interaction histories to grow: multiple rounds of model generation and environmental feedback become interleaved, forming complex, evolving contexts that are not adequately captured by existing synthetic, text-only long-context benchmarks. To address this gap, we introduce Agentic LongBench, evaluating frontier LLMs on diverse agentic tasks with long and easily extensible context. Additionally, we study two primary test-time scaling paradigms on our benchmark—parallel scaling and sequential scaling—to characterize model scaling behavior. By extending context lengths up to 256K tokens and sampling up to eight trajectories, we find that allocating compute to self-refinement yields limited performance gains, whereas increasing the number of samples leads to substantial improvements. Additionally, we provide two further analyses—attention analysis and agent design analysis—to demonstrate how our benchmark can be used to study design factors that influence models’ agentic capabilities.

% \begin{abstract}

True agentic ability requires solving long-horizon agentic tasks without reliance on domain-specific hints. However, existing benchmarks often suffer from domain leakage, where provided tool descriptions inadvertently introduce prior knowledge bias. To address this, we present Omni AgenticBench, a benchmark spanning four domains that explicitly removes such leakage to enable fair evaluation. Additionaly, we systematically investigate test-time scaling strategies, revealing distinct and often counterintuitive performance trends across models. Beyond standard pass@k metrics, we introduce a self-choice evaluation, where models identify the correct final answer from multiple generated candidates. Our results expose a substantial gap between the models' solution space and their self-cognition space. Finally, we analyze attention mechanisms of recent linear and sparse architectures under agentic settings. All tasks, code, and data are publicly available.

\end{abstract}


\section{Introduction}

% \cx{make this a benchmark paper? then the studies are insights we have with the benchmark. title is not reflecting this.}

Developing general-purpose agents powered by Large Language Models (LLMs) has emerged as a primary focus of current AI research\cite{yao2022react, liu2023agentbench, schick2023toolformer}. A defining characteristic of agentic tasks is their reliance on extensive context; predefined tool descriptions, complex user queries, model-generated reasoning, and environmental feedback collectively form intricate, long-context, multi-turn interaction histories\cite{qin2023toolllm,zhou2023webarena,yang2024sweagent}. Consequently, the performance of these agents is inextricably linked to the model’s ability to reason over both tool specifications and long-horizon information.

While agentic benchmarks are well-established for evaluating frontier models, their evaluation protocols often suffer from a critical shortcoming. To achieve true generality, future agents must solve problems without prior assumptions regarding the task content\cite{liang2025sweillusion} or category. However, existing benchmarks typically provide only domain-specific tool descriptions for a given task, resulting in a constrained and narrow evaluation environment. For instance, SWE-bench\cite{jimenez2023swebench} provides only bash commands tools in their policy prompts to facilitate software engineering tasks, while BrowserComp\cite{wei2025browsecomp} defines only a search tool to assist with web navigation without other tools' interference. We argue that these predefined toolsets unintentionally signal the task's domain to the agent—a phenomenon we term \textbf{``domain leakage''}—thereby simplifying the problem and failing to test the agent's ability to operate in a truly open-ended environment.

In this paper, we bridge this gap by introducing Omni AgenticBench, a benchmark designed to comprehensively evaluate agentic capabilities across diverse categories—including search, coding, reasoning, and tool-use—within a unified toolset framework. We began by auditing widely used agent benchmarks and evaluating frontier open-source models to ensure our task selection was both high-quality and representative of state-of-the-art challenges. To eliminate domain leakage, we consolidated tool definitions and descriptions from all domains into a single, unified library. We then replaced domain-specific policy prompts and constrained toolsets with this global library and a set of general instructions. This configuration allows us to decouple an agent's true reasoning ability from the performance gains provided by domain-specific hints. Our comprehensive evaluation of ten leading LLMs reveals a significant performance degradation for most agents when prior domain hints are removed. Interestingly, a subset of models maintains or even marginally improves performance, suggesting a higher degree of robust, generalizable agency. Furthermore, because our merged "Omni-toolset" accumulates to approximately 64K tokens, the benchmark inherently serves as a rigorous test of long-context utilization. By analyzing the correlation between Omni AgenticBench and traditional long-context benchmarks—such as Needle In A Haystack\cite{kamradt_needlehaystack_2023} and long-document QA\cite{bai2025longbenchv2}—we demonstrate that high performance on static long-context tasks does not necessarily transfer to the dynamic requirements of agentic workflows.

% We evaluate not only frontier large language models (LLMs) on Agentic LongBench to ground their long-context abilities in agentic scenarios, but also include recent innovations in attention design—such as Qwen3-Next and DeepSeek-v3.2—to reveal how modifications to the attention mechanism affect performance in stressful real-world cases. Although the adoption of sparse attention and linear-attention mixture architectures aims for higher efficiency and reduced memory bottlenecks, their performance still lags significantly behind established models like GPT-5 and Claude 4.5 Sonnet, highlighting a persistent performance gap for these attention variants. Furthermore, we evaluate these models on existing representative long-context benchmarks and compare their score correlations with our findings. The low correlation between the two suggests that single-turn long-context proficiency does not directly transfer to agentic settings. To provide a more robust assessment, we introduce the Consistency-Accuracy Index (CAI), which balances raw performance with stability. Our findings show that, with the notable exception of GPT-5, most models exhibit unstable ranking trends, struggling to maintain high mean accuracy alongside behavioral consistency.

% While the context window of backbone LLMs has scaled to millions of tokens, effectively eliciting and evaluating their long-context potential remains an open challenge. Existing benchmarks designed to assess long-context abilities suffer from two critical limitations: (1) Task Distribution Mismatch: Traditional benchmarks often rely on long-document Question Answering (QA) or summarization, simply scaling the input size without incorporating the core elements of agentic workflows—namely, dynamic environments, tool-use, and nuanced user intents. (2) Structural Mismatch: Current benchmarks are predominantly single-turn. This format fails to capture the complexity of agentic tasks, where the long context is composed of the model’s own historical actions, such as command executions, intermediate summaries, and self-reflections. These recursive dependencies are fundamentally absent in single-turn evaluation datasets, therefore failed to mimic real world agent tasks. Furthermore, current agent related benchmarks often suffer from scope constraints: they either lack tool integration, which artificially restricts the model's action space, or focus exclusively on niche domains such as machine learning engineering or gaming. Consequently, they fail to provide a comprehensive analysis of the relationship between task complexity and context length.

% In this paper, we address these gaps by introducing Agentic LongBench, a benchmark specifically curated to comprehensively evaluate agentic capabilities across diverse task categories, including search, coding, reasoning, and tool-use domains. We evaluate not only frontier large language models (LLMs) on Agentic LongBench to ground their long-context abilities in agentic scenarios, but also include recent innovations in attention design—such as Qwen3-Next and DeepSeek-v3.2—to reveal how modifications to the attention mechanism affect performance in stressful real-world cases. Although the adoption of sparse attention and linear-attention mixture architectures aims for higher efficiency and reduced memory bottlenecks, their performance still lags significantly behind established models like GPT-5 and Claude 4.5 Sonnet, highlighting a persistent performance gap for these attention variants. Furthermore, we evaluate these models on existing representative long-context benchmarks and compare their score correlations with our findings. The low correlation between the two suggests that single-turn long-context proficiency does not directly transfer to agentic settings. To provide a more robust assessment, we introduce the Consistency-Accuracy Index (CAI), which balances raw performance with stability. Our findings show that, with the notable exception of GPT-5, most models exhibit unstable ranking trends, struggling to maintain high mean accuracy alongside behavioral consistency.

The multi-turn interaction format of agentic tasks make it easy for agents to allocate more compuation and enable test-time scaling: by forcing more turns and encouraging agents to reflect, we expect models to utilize their reasoning abilities to revise answers, retrieve missing information from the context, or develop new reasoning paths. While test-time scaling trend has been well studied for pure reasoning tasks, the behavior of agent scaling remains under-explored. We provide a systematic study of two primary scaling methods in agentic settings: (1) parallel Scaling: independently sampling $K$ trajectories and calculating accuracy via the best@$K$ metric. (2) sequential scaling: monitoring the model's "closing intent" and manually intervening with an additional turn of encouragement to prompt further reflection and thought. Our results across five models demonstrate that while parallel scaling follows expected improvement trends, sequential scaling exhibits the opposite phenomenon: nearly all models suffer performance degradation even with four times the computation. In some cases, sequential scaling performance falls below the baseline evaluation. Further analysis reveals that current LLMs either struggle to identify their own errors buried within the context or fail to maintain their initial correct answers, eventually overturning them. This indicates that even frontier models struggle to reason effectively over raw, accumulated trajectory contexts.

While agents often achieve high best@$K$ scores, this metric merely indicates that a human can verify the task and that a correct solution exists within the agent’s solution space. It provides no guarantee that the agent can identify that correct generation. For parallel scaling to be effective, an agent must not only sample the correct answer but also "recognize" it. This capability is essential for agents to improve autonomously, particularly in non-verifiable tasks. To address this, we move beyond simple best@$K$ metrics to evaluate "self-selection"—a setting where the agent must evaluate its own candidates and select a final answer. We examine two selection methodologies: point-wise, where the agent evaluates generations individually, and pair-wise, where the agent compares two generations at a time using an iterative approach similar to bubble sort. Under this framework, a task is successful only if a correct generation is both produced and selected by the model. Our results reveal a significant gap between best@$K$ results and self-selection accuracy, highlighting a critical need to improve agent self-cognition to better align a model's judgment with its internal solution space.

% Consequently, agentic tasks inherently integrate long-context processing with reasoning. Their features to easily allocate additional computation makes them an ideal testbed for analyzing test-time scaling behavior. While test-time scaling has proven effective for pure reasoning tasks, its application to agentic workflows remains under-explored. We provide a systematic study of two primary scaling methods in agentic settings: (1) Parallel Scaling: Independently sampling $K$ trajectories and calculating accuracy via the Best@$K$ metric. (2) Sequential Scaling: Monitoring the model's "closing intent" and manually intervening with an additional turn of encouragement to prompt further reflection and thought. Our results across five models demonstrate that while parallel scaling follows expected improvement trends, sequential scaling exhibits the opposite phenomenon: nearly all models suffer performance degradation even with four times the computation. In some cases, sequential scaling performance falls below the baseline evaluation. Further analysis reveals that current LLMs either struggle to identify their own errors buried within the context or fail to maintain their initial correct answers, eventually overturning them. This indicates that even frontier models struggle to reason effectively over raw, accumulated trajectory contexts.

Finally, we demonstrate that Omni AgenticBench serves as a powerful diagnostic testbed through an attention analysis. Our study covers standard scaled dot-product attention as well as architectural innovations such as sparse and linear attention. We utilize a reasoning-behavior framework to identify critical reasoning spans within the input text, subsequently computing accumulated attention over these spans to extract the top-$k$ most influential tokens. By calculating the top-$k$ token overlap across layers (inter-layer) and heads (intra-layer), we quantify the functional diversity and redundancy inherent in different architectures. We further characterize the "effective receptive field" of these components by computing the average attention distance.  Finally, we provide case studies by mapping top-$k$ attention tokens back to the input text, revealing the specific contextual information that triggers reasoning behaviors. In summary, our contributions are:

% Finally, we demonstrate that Omni AgenticBench can serve as a powerful diagnostic testbed by providing an additional attention patterns analysis. We adopt a reasoning-behavior framework to identify key reasoning spans in the text first. Then, by computing accumulated attention over these spans and mapping top-$k$ scores back to tokens, we investigate which contextual information triggers reasoning. While some high-attention tokens correspond to relevant content (e.g., location tokens triggering geographical query modifications), many high-scoring tokens lack semantic meaning, reflecting classic issues such as attention sinks and local attention bias. In our context engineering analysis, we contrast task-specific tool availability with an "Omni-setting" (providing all available tools regardless of the task). This removes the prior bias of pre-filtered toolsets. Our results show a nearly universal performance drop in the Omni-setting, challenging the community to build agents that are truly environment-agnostic. In summary, our contributions are:



\begin{itemize}
    \item \textbf{Omni AgenticBench Framework:} We introduce Omni AgenticBench, a novel evaluation framework designed to eliminate the prevalent issue of \textbf{domain leakage}. By decoupling domain-specific hints from task execution, it provides a rigorous and unbiased assessment of an agent’s ability to reason over long-horizon contexts across diverse domains.
    \item \textbf{Characterization of Test-Time Scaling:} We present a systematic study of test-time scaling behaviors in agentic settings. Our analysis identifies a critical bottleneck in model self-cognition: while parallel scaling expands the solution space, models frequently fail to identify the correct answer among their samples. We also reveal that sequential scaling often leads to performance degradation in complex agentic workflows.
    \item \textbf{Mechanistic Analysis across Attention Architectures:} We utilize Omni AgenticBench as a diagnostic testbed to uncover distinct attention patterns across various architectures, including linear and sparse attention mechanisms. Through our reasoning-behavior framework, we quantify the functional diversity and "effective receptive fields" of different layers and heads, mapping specific contextual triggers to reasoning behaviors.
\end{itemize}

% First, mechanistic analysis reveals a weak correlation between high-attention tokens and semantic reasoning, often overshadowed by attention sinks and local bias. Second, by introducing the "Omni-setting," we demonstrate a universal performance drop when agents are not provided with pre-filtered toolsets, highlighting the need for more environment-agnostic agent architectures.
% \section{Introduction}

\begin{figure*}[t]
    \centering
    \includegraphics[width=0.8\linewidth]{figs/intro_figure_radar_1x3.pdf}
    \caption{\textbf{Evaluating agents under a realistic user-interaction scenario.} \textbf{Left}: When evaluated within Omni AgenticBench framework, GPT-5 exhibits a substantial performance drop compared to static evaluations with fixed, pre-specified contexts. \textbf{Right above}: Sequential test-time scaling, allocating additional computation through extended interaction histories, often leads to unstable performance and can even degrade agent accuracy, despite increased compute. \textbf{Right bottom}: Although correct solutions increasingly appear in the model’s generation space as the number of samples grows (blue, best@k), agents frequently fail to identify these solutions (red line), revealing a pronounced verification gap between generation and self-cognition.}
    \label{fig:intro_figure}
\end{figure*}

Developing general-purpose agents powered by Large Language Models (LLMs) has emerged as a central focus of contemporary AI research. These agents, equipped with real-world tools and strong reasoning capabilities inherited from LLMs, are deployed to interact with human users and solve a wide variety of complex, open-ended requests. To evaluate such capabilities, numerous agentic benchmarks have been proposed, covering major task categories including coding, web search, tool calling, and computer-use. These benchmark designers typically curate distinct, difficulty-varied questions to comprehensively test agent performance within specific, isolated environments.

However, the complexity of real-world user queries stems not only from content diversity, but also from the absence of explicit task-category assumptions during user–agent interaction. Existing agentic benchmarks typically evaluate agents under predefined, domain-specific contexts with restricted toolsets, which are well suited for assessing specialized agents within known task settings. For example, SWE-bench exposes only Bash-related tools to facilitate software engineering tasks, while BrowserComp defines a single search tool to support web navigation. In contrast, general agents operate without prior knowledge of a query’s domain and must first infer the underlying task type before selecting appropriate tools. As a result, existing evaluation settings do not explicitly model this form of task-category uncertainty, leaving the general-agent setting underexplored in current benchmarks.\cx{this paragraph can use more revision, high level real world need argument, and also focus more on this work, less than swebench etc, also argue that not tast uncertainty, but composition of different skills and tools in on complicated task that span across search, coding, and math, etc.}

In this paper, we bridge this gap by introducing Omni AgenticBench, a benchmark for evaluating general-purpose agents across diverse scenarios—including search, coding, reasoning, and tool use—under a unified toolset framework that more closely reflects real-world interactions with users. We began by auditing widely used benchmarks to ensure our task selection remains high-quality and representative of state-of-the-art challenges. \cx{missing a general overview of our benchmark, use the previous sentence's spacae for that?} We consolidated tool definitions and descriptions from all domains into a single, unified toolsets. By replacing domain-specific policy prompts with a global toolset and a general policy, we evaluate agents in a setting that more closely mirrors real-world interactions, where task domains are not known in advance. Our evaluation of ten leading LLMs under this setting shows a significant performance degradation when moving from domain-aware configurations to a general-agent context. \cx{here can discuss different properties of differenet llms}


A defining aspect of agentic capability lies in maintaining and reasoning over long, evolving contexts. First, the unified toolset consists of numerous tool descriptions, parameters, and general policy instructions, which together form a substantial long-context input. Second, multi-turn interactions between agents, users, and environments continuously expand the context through accumulated trajectories.  Rather than treating long-context capability as a static property that is commonly evaluated by previous benchmarks such as Needle In A Haystack or long-document QA, we study how agents behave when operating under long and dynamically growing contexts. Specifically, we use \textbf{test-time scaling} as a practical way to examine whether agents can reason consistently, make stable decisions, and effectively use information as the context length increases. We conduct a systematic study of two primary scaling methods: (1) parallel scaling, which independently samples $K$ trajectories; and. (2) sequential scaling, which introduces additional interaction turns to encourage iterative reflection. Our results across five models demonstrate that while parallel scaling yields expected improvement, sequential scaling exhibits the opposite trend: nearly all models suffer performance degradation despite increased computation. In some cases, performance falls below the single-pass baseline, suggesting that even frontier models struggle to reason effectively over raw, accumulated trajectory contexts. We also demonstrate that high performance on static long-context tasks does not necessarily transfer to the dynamic, multi-turn requirements of agentic tasks.
% As a result, agents must reason over extended histories that include tool specifications, prior decisions, and intermediate results.

While multiple sampling often achieve high accuracy, this merely indicates that a correct solution exists somewhere within the sampled solution space. It does not guarantee that the agent can identify that solution. In real-world interactions, an agent cannot present $K$ responses simultaneously and must instead commit to a single answer. Consequently, for parallel scaling to be practically effective, an agent must not only generate correct candidates but also recognize the best one. This capability is critical for autonomous agents, particularly in non-verifiable settings. To evaluate this ability, we move beyond best@$K$ and introduce self-selection, where the agent must assess its own candidates and select a final answer. Our results reveal a substantial gap between best@$K$ performance and self-selection accuracy, highlighting a fundamental limitation in agent self- verification. \cx{you are presenting the work sequentially as separate parts in the order of them appear in the paper, this is wrong. Really present as the introduction structure: what we do and how we do it, as a general benchmark that covers many important aspects of general agents, then discuss the overal findings (all together), insights (all together), future potentials (all together)}

To explain the performance degradation observed under the general-agent setting, we conduct a targeted analysis of agent failure modes. First, we analyze agent behavior from the perspective of cross-domain tool selection. Under the unified toolset, agents must infer both task type and appropriate tool usage. By comparing trajectories under the original benchmark settings and the general-agent setting, we identify three representative patterns: (i) agents fail after invoking tools from incorrect domains; (ii) agents initially explore incorrect-domain tools but recover by switching to the correct tool and succeed; and (iii) agents select the correct tool throughout yet still fail under the general setting. Notably, performance on certain search benchmarks improves under the unified toolset, indicating that benchmark-specific tool restrictions may limit the observable capability of current agents. Second, we analyze outcome transitions under sequential test-time scaling by tracking how predictions evolve with additional computation. We measure how often additional interaction steps turn an initially incorrect solution into a correct one, and how often they instead cause a previously correct solution to fail, helping explain why sequential scaling frequently underperforms despite increased computation.

% Together, these analyses provide concrete explanations for the observed failures in realistic, long-horizon general-agent settings.

% Finally, the long-context nature of agentic tasks has motivated the development of alternative attention mechanisms to mitigate the $\mathcal{O}(n^2)$ complexity of standard scaled dot-product attention. In our evaluation, we include production models such as Qwen3-Next and DeepSeek-v3.2 to study the effects of sparse and linear attention mechanisms in agentic settings. Beyond aggregate performance, we analyze attention patterns along agent trajectories to examine the relationship between historical context and reasoning behavior. By extracting top-$k$ attention tokens associated with key reasoning steps, we characterize attention distance and token overlap patterns unique to agentic workflows. While certain similarities with non-agentic tasks per   sist, we observe notable differences driven by long-horizon interaction and tool usage. We further present qualitative case studies that map top-$k$ attention tokens back to the input context, providing interpretable insights into how specific historical information triggers agent reasoning.

In summary, our contributions are:

% \begin{itemize}
%     \item \textbf{Omni AgenticBench:} We introduce Omni AgenticBench, a novel evaluation framework designed to eliminate the prevalent issue of \textbf{domain leakage}. By decoupling domain-specific hints from task execution, it provides a rigorous and unbiased assessment of an agent’s ability to reason over long-horizon contexts across diverse domains.
%     \item \textbf{Characterization of Test-Time Scaling:} We present a systematic study of test-time scaling behaviors in agentic settings. Our analysis identifies a critical bottleneck in model self-cognition: while parallel scaling expands the solution space, models frequently fail to identify the correct answer among their samples. We also reveal that sequential scaling often leads to performance degradation in complex agentic workflows.
%     \item \textbf{Mechanistic Analysis across Attention Architectures:} We utilize Omni AgenticBench as a diagnostic testbed to uncover distinct attention patterns across various architectures, including linear and sparse attention mechanisms. Through our reasoning-behavior framework, we quantify the functional diversity and effective attention distance of different layers and heads, mapping specific contextual triggers to reasoning behaviors.
% \end{itemize}

\begin{itemize}
    \item \textbf{Omni AgenticBench for Real-World Agent Evaluation.} 
    We introduce Omni AgenticBench, a benchmark designed to narrow the gap between existing agent evaluations and real-world user interactions. By providing unified toolsets and instructions, the benchmark evaluates whether agents can infer task intent, select appropriate tools, and reason over long, evolving interaction contexts in open-ended settings.

    \item \textbf{Systematic Test-Time Scaling Behaviors of General Agents} 
    We analyze how agents reason over long, evolving contexts using test-time scaling as a diagnostic tool. By comparing parallel and sequential scaling, we show that sequential scaling often degrades performance as interaction history grows, exposing limitations in agents’ ability to maintain stable reasoning over accumulated contexts. We further identify a self-cognition bottleneck: although parallel scaling expands the solution space, agents frequently fail to recognize correct solutions among their own samples, limiting their effectiveness in real-world interactions.

    \item \textbf{Failure Analysis of General-Agent Behavior.} 
    We analyze agent failure modes under unified toolsets, focusing on cross-domain tool selection errors and performance instability under sequential test-time scaling. Our results show that agents are often misled by irrelevant tools and that longer interaction histories more frequently degrade correct solutions than fix incorrect ones, explaining the observed performance drop in general-agent settings.

\end{itemize}


\cx{intro has the right element and the story reads coherent now. need revision on reorg the content to fit in the right structure of introduction, polish the language to be more concise (now it is too long too), and twist some argument and statement to be coherent (and accurate)}

\cx{we need related work section. perhaps can be the last section before conclusion. need to cover single agentic benchmarks, general agent development, and test time scaling work}

% Furthermore, as our merged "Omni-toolset" exceeds 64K tokens, the benchmark inherently serves as a rigorous test of long-context utilization. By analyzing the correlation between Omni AgenticBench and traditional long-context benchmarks such as Needle In A Haystack and long-document QA, we demonstrate that high performance on static long-context tasks does not necessarily transfer to the dynamic, multi-turn requirements of agentic tasks.


% We examine two selection methodologies: point-wise, where the agent evaluates generations individually, and pair-wise, where the agent compares two generations at a time using an iterative approach similar to bubble sort. 
\section{Introduction}

\begin{figure*}[t]
    \centering
    \begin{subfigure}[t]{0.38\linewidth}
        \centering
        \includegraphics[width=\linewidth]{figs/intro_figure_a_radar.pdf}
        \caption{Performace comparsion.}
        \label{fig:intro_a}
    \end{subfigure}
    \hfill
    \begin{subfigure}[t]{0.3\linewidth}
        \centering
        \includegraphics[width=\linewidth]{figs/intro_figure_b_scaling.pdf}
        \caption{Sequential test-time scaling.}
        \label{fig:intro_b}
    \end{subfigure}
    \hfill
    \begin{subfigure}[t]{0.3\linewidth}
        \centering
        \includegraphics[width=\linewidth]{figs/intro_figure_c_gap.pdf}
        \caption{Parallel test-time scaling.}
        \label{fig:intro_c}
    \end{subfigure}

    \caption{\textbf{Evaluating general LLM agents under a realistic user-interaction scenario.}
    \textbf{A}: GPT-5's performance drop under General AgentBench compared to static, domain-specified evaluation.
    \textbf{B}: Sequential test-time scaling via longer interaction histories can lead to unstable or degraded performance.
    \textbf{C}: While correct solutions increasingly appear in the generation space (past@$K$), agents often fail to select them, revealing a verification gap.}
    \label{fig:intro_figure}
\end{figure*}


Agents powered by large language models (LLMs) are at a turning point, transitioning from domain-specific \cite{liu2023fingpt,yang2024sweagent, yue2024clinicalagent} to general-purpose \cite{xi2023rise, luo2025large}. Real-world user requests are often open-ended and require LLM agents to operate end-to-end through planning \cite{wang2023plan, erdogan2025planandact}, reasoning \cite{Wei2022CoT, yao2022react, parmar2025plangen}, and tool use \cite{schick2023toolformer, patil2024gorilla} under uncertain and evolving conditions. A capable general agent is expected to compose multiple skills and tools (e.g., search, coding, computation, and MCP APIs) to handle the diversity of realistic requests\cite{claudeskills}, while exhibiting effective test-time scaling abilities to address increasing task complexity and enhance response quality \cite{Wang2022SelfConsistency, brown2024large, snell2024scaling}. This shift raises an important evaluation gap: beyond asking \emph{“can the model solve a task,”} we must assess whether agents can \emph{infer user intent, select specific tools, and scale up their performance under a unified evaluation framework across diverse domains}.

Existing agentic benchmarks typically evaluate LLM agents in domain-specific settings, where the environment and available toolsets are explicitly designed for a particular task category (e.g., software engineering with a Docker environment and terminal tools \cite{jimenez2023swebench, Aleithan2024SWEBenchPlus}, or web navigation with a browsing interface \cite{zhou2023webarena, he2024webvoyager, wei2025browsecomp}). Conversely, actual user interactions are rarely so constrained; they typically involve multi-turn, open-ended requests spanning disparate domains, requiring agents to remain ready across a broad toolset to handle unpredictable queries. As a result, while current benchmarks are informative for domain-specific agent development, they may not fully capture the demands of real-world usage and can overestimate robustness under realistic, multi-domain conditions.


% A defining challenge in this setting is reasoning over long, evolving contexts. The unified toolset itself contributes substantial context through tool descriptions, parameters, and policies, and multi-turn trajectories further expand context via tool outputs and the agent’s own prior decisions.To study how agents behave under such dynamically growing contexts, we use test-time scaling as a practical lens to examine whether agents can stabilize or even improve as context length increases.

In this paper, we address this gap by introducing General AgentBench, a benchmark designed to evaluate general-purpose agents across diverse scenarios under a unified framework that more closely reflects real-world user interactions. We consolidate tools from all domains into a shared interface that is consistently exposed to evaluated agents across different tasks, while domain-specific environments and implementations remain hidden. We evaluate ten leading LLM-based agents, each of which must first interpret the user request, then choose suitable tools from a large and diverse tool pool, and iteratively interact with the environment until producing a final response. 

For complex requests that exceed the capabilities of a short interaction horizon, agents can benefit from increased inference-time computation—a strategy known as test-time scaling and extensively studied in the context of non-agentic reasoning \cite{cobbe2021training, zelikman2024quiet, guo2025deepseek}. We focus on two primary scaling strategies: (1) sequential scaling, which extends interaction histories to support continued reasoning, reflection, and exploration; and (2) parallel scaling, which independently samples $K$ candidate trajectories and selects a single candidate to return. More concretely, effective parallel scaling requires agents not only to \textbf{generate} correct solutions, but also to reliably \textbf{identify and choose} the correct one, since real-world agents cannot present multiple responses simultaneously. Together, these settings enable a systematic study of test-time scaling behaviors in general LLM agents.

Our results lead to three key conclusions. (1) Across ten leading LLMs, we observe a substantial performance drop when moving from domain-specific configurations to the general-agent setting (Figure~\ref{fig:intro_a}), with pronounced differences in robustness across model families. Among them, Claude exhibits the strongest robustness, while most other models experience performance drops of approximately 30\%. (2) Sequential scaling exhibits an effective context length ceiling (\textbf{context ceiling}): while performance improves within a modest range of additional interaction turns, it often fluctuates or degrades thereafter. This suggests that the accumulated history eventually overwhelms the agent's reasoning capacity, leading to instability in long-horizon tasks. (Figure~\ref{fig:intro_b}). (3) Although parallel scaling increases the theoretical upper bound of performance (past@$K$), we consistently observe a gap between this upper bound and self-choice accuracy, revealing a substantial \textbf{verification gap} that ultimately limits achievable performance in realistic settings (Figure~\ref{fig:intro_c}). In summary, our contributions are:

% We further conduct detailed analyses to reveal distinct behaviors of general agents on our benchmark, including cross-domain tool-use selection, mechanisms behind sequential-scaling degradation, and comparisons of attention distance between linear and full attention mechanisms.

\begin{itemize}
\item \textbf{General AgentBench for Realistic Evaluation.} 
We introduce General AgentBench, a benchmark for evaluating whether agents can compose multiple skills and tools to solve open-ended requests from diverse domains under a unified framework, more closely reflecting real-world user interactions.

\item \textbf{Sequential Test-time Scaling in General Agents.} 
We study the sequential test-time scaling behavior of general agents, showing that performance improvements are bounded by an effective context ceiling, beyond which additional computation often leads to instability and performance degradation. This inherent point varies across models and domains.

\item \textbf{Parallel Test-time Scaling.} 
We analyze parallel test-time scaling and show that, despite increasing the theoretical performance upper bound (past@$K$), its practical gains are limited by a verification gap between generation and model self-choice accuracy.
\end{itemize}




\section{General AgentBench}

In this section, we introduce the construction of General AgentBench (Section~\ref{2.1}) and the unified evaluation framework (Section~\ref{2.2} ). Detailed prompt templates, tool specifications, and the unified policy are deferred to the Appendix~\ref{appendix:prompt}.

% \subsection{Domains and Sources}\label{2.1}
% Omni AgenticBench covers four major task categories: \textbf{Coding}, \textbf{Search}, \textbf{Tool-use}, and \textbf{Reason}. We select data sources that are widely used as reference benchmarks when new proprietary or open-weight models are released to ensure the quality. Table \ref{tab:composition} reports the statistics of our benchmark. 

% \cx{need to discuss what the role of data sources are for us}\xiaochuan{check}

% \cx{always use section intro to keep reader organized, you have space else where}

\subsection{Domains and Sources}\label{2.1}

Our benchmark spans four task domains: \textbf{Coding}, \textbf{Search}, \textbf{Tool-use}, and \textbf{Reason}. These domains reflect common real-world applications such as software engineering, information seeking, service workflows, and analytical reasoning, positioning General AgentBench as an initial step toward evaluating general-purpose agents in open-ended and unified settings. Table~\ref{tab:composition} summarizes the benchmark composition.

\begin{table}[h]
\centering
\small
\caption{Composition of General AgentBench}
\label{tab:composition}
\resizebox{0.9\linewidth}{!}{%
\begin{tabular}{llrr}
\toprule
\textbf{Domain} & \textbf{Dataset} & \textbf{Original} & \textbf{Sampled} \\
\midrule
\multirow{2}{*}{Search}
  & BrowseComp   & 1266 & 124\\
  & WebVoyager   & 643  & 65 \\
\midrule
\multirow{2}{*}{Coding}
  & SWE-Bench Verified & 500 & 50 \\
  & Terminal-Bench     & 230 & 80 \\
\midrule
Reason
  & MathHay            & 602 & 75 \\
\midrule
\multirow{2}{*}{Tool-Calling}
  & Tau2-Bench         & 278 & 50 \\
  & MCP-Bench          & 104 & 52 \\
\bottomrule
\end{tabular}%
}
\end{table}

\paragraph{Coding}
We include tasks from SWE-Bench Verified \cite{openai_swebench_verified_2024} and Terminal Bench, which evaluate an agent’s ability to analyze production-level software issues, reason over long instructions, and iteratively interact with execution environments to reach a correct final state.

\paragraph{Search}
The search domain includes tasks from BrowseComp \cite{wei2025browsecomp} and WebVoyager \cite{he2024webvoyager}. These benchmarks assess an agent’s ability to identify missing information, decide when additional search steps are needed, and navigate long, evolving web contexts, going beyond static retrieval or single-turn question answering.

\paragraph{Tool-use}
For tool-use, we adopt Tau2-Bench \cite{barres2025tau2} and MCP-Bench \cite{wang2025mcpbench}, both of which provide rich tool suites requiring models to select, invoke, and coordinate multiple tools. These tasks emphasize structured tool calling and multi-step planning in realistic service and workflow scenarios.

\paragraph{Reason}
For long-context reasoning, we use MathHay \cite{wang2024mathhay}, which constructs queries by embedding relevant mathematical documents into noisy long-context haystacks. This benchmark isolates sustained reasoning over long inputs without relying on external tool execution, complementing the other domains.

% \cx{the discussion of these four existing categoties and benchmarks can be much shorter, more details can be moved to appendix, just general introduction are needed (and some highlight of why each task is good). Thus we have more space for our stuff.}\xiaochuan{check}

\begin{figure}[h]
    \centering
    \includegraphics[width=0.93\linewidth]{figs/bench_construction.pdf}
    \caption{\textbf{Illustration of how General AgentBench covers a wide range of task categories while providing a unified interface to simulate real-world user interactions.} The green region indicates the specific task currently being handled by the agent (e.g., a search task). Orange boxes denote other clients and servers that remain active and responsive but are not directly involved in the current interaction. Red indicates that other domain-specific data are excluded.}
    \label{fig:omni-setting}
\end{figure}

\begin{table*}[t]
\centering
\small
\caption{Main results on \textbf{General AgentBench}. Benchmarks are grouped by domain. \textbf{Avg.} denotes the mean score across all available benchmarks for each model. Bold indicates the best score.}
\label{tab:omni_only_domain_avg}
\setlength{\tabcolsep}{3.5pt}
\begin{tabular}{@{} l rr rr r rr r @{}}
\toprule
\multirow{2}{*}{\textbf{Models}} &
\multicolumn{2}{c}{\textbf{Search}} &
\multicolumn{2}{c}{\textbf{Code}} &
\textbf{Reason} &
\multicolumn{2}{c}{\textbf{Tool-use}} &
\multirow{2}{*}{\textbf{Avg.}} \\
\cmidrule(lr){2-3} \cmidrule(lr){4-5} \cmidrule(lr){6-6} \cmidrule(lr){7-8}
& \textbf{BrowseComp} & \textbf{WebVoyager}
& \textbf{SWE-Bench} & \textbf{Terminal-Bench}
& \textbf{MathHay}
& \textbf{Tau2-Bench} & \textbf{MCP-Bench}
& \\
\midrule

\textit{Open-Source} \\

GPT-OSS-120B
& 4.0 & 27.7 & 12.0 & 6.3 & 38.7 & 26.0 & 63.3 & 25.4 \\

Qwen3-235B-A22B
& 8.9 & 30.8 & 20.4 & 23.8 & 32.0 & 38.3 & 66.1 & 31.5\\

Qwen3-Next
& 10.5 & 35.4 & 18.0 & 8.8 & 42.0 & 48.9 & 64.6 & 32.6 \\

DeepSeek-V3.2
& 19.4 & 46.2 & 31.8 & 22.2  & 33.3 &  \bfseries
 54.0 &  66.0 & 39.0 \\

DeepSeek-R1
& 9.7 & 43.1 & 14.0 & 8.8 & 46.7 & 17.1 & 62.2 & 28.8 \\

\hline
% \addlinespace[0.5em]
\textit{Proprietary} \\

Gemini 2.5-Flash
& 6.5 & 32.3 & 14.0 & 20.0 & 36.0 & 38.3 & 66.6 & 30.5 \\

Gemini 2.5-Pro
& 8.9 & 46.2 & 26.0 & 27.5 & 24.0 & 46.0 & 67.2 & 35.1 \\

Claude Haiku 4.5
& 17.7 & 47.7 & \bfseries
 56.0 & 25.0 & 34.7 & 44.0 & 69.0 & 42.0 \\

Claude Sonnet 4.5
& 23.1 & 56.9 & 54.0 & \bfseries
 45.0 & 36.0 & 48.0 & \bfseries
 72.9 & \bfseries 48.0 \\

GPT-5
& \bfseries
 27.4 & \bfseries
 61.5 & 36.0 & 41.3 & \bfseries
 64.0 & 32.0 & 59.1 & 45.9 \\

\bottomrule
\end{tabular}
\end{table*}


\subsection{Unified Realistic Evaluation Framework}\label{2.2}

% These choices reflect three fundamental properties of real-world agent usage: cross-domain task diversity, comprehensive skill requirements, and dynamically evolving multi-turn interactions.

To support realistic evaluation of general LLM agents, we design a unified framework that exposes all tasks and tools through a shared interaction interface. These choices reflect three fundamental properties of real-world agent usage: cross-domain task diversity, comprehensive skill requirements, and dynamically evolving multi-turn interactions. An overview of the framework is illustrated in Figure~\ref{fig:omni-setting}.

\paragraph{Unified tool interface.}
In practical deployments, agents must select appropriate tools from a large pool without prior knowledge of task domains. To reflect this setting, we adopt the Model Context Protocol (MCP) \cite{mcp} as the backbone of our framework. Each benchmark environment is instantiated as an MCP server, while all servers are centrally managed by a unified Host. The Host maintains a global tool registry that records all available tools and their corresponding server mappings, presenting the agent with a single, unified tool space across all domains.

\paragraph{Centralized interaction abstraction.}
The Host serves as the sole interaction interface for the general agent, abstracting away individual benchmark implementations. When the agent invokes a tool, the Host resolves the call via the tool registry and routes the request to the appropriate server for execution. 

% This abstraction prevents the agent from directly accessing domain-specific environments and eliminates implicit domain cues, encouraging agents to infer task structure and tool relevance purely from interaction context.

\paragraph{Evolving interaction context.}
Because all tools and benchmark environments are exposed simultaneously, the unified tool descriptions alone can span tens of thousands of tokens. When combined with user queries and accumulated multi-turn interaction histories, the resulting context naturally grows into the long-context regime. In this setting, agents must reason over heterogeneous information sources, including task instructions, tool documentation, execution feedback, and their own prior decisions. This distinguishes agentic interaction from many existing long-context benchmarks that focus on static, single-turn question answering or summarization with short outputs. We provide further long-context study in Appendix ~\ref{appendix:long_compare}

\paragraph{Execution process.}
For each evaluation instance, the framework provides it to the agent together with the unified policy and toolset as the context. All benchmark servers (e.g., Docker-based environments in the coding domain) are instantiated simultaneously and remain idle while awaiting requests from the agent. When the agent issues a tool call, the Host routes the request to the corresponding server, executes the tool, and returns the result in a unified response format. Tool execution is decoupled from task type: even if a task is search-oriented, code-related tool calls can still be executed by the environment, returning valid outputs despite having no direct relevance to the final solution. This design intentionally exposes the agent to a realistic setting where incorrect or irrelevant tool usage remains possible.

The agent interacts with the framework over multiple turns until producing a final answer. During interaction, we monitor execution signals (e.g., terminal outputs) and regulate the interaction budget, enabling additional computation or extended reasoning when applicable (Section~\ref{4.1}). The final answer is then forwarded to the corresponding benchmark server for correctness evaluation.

% Together, these properties differentiate Omni AgenticBench from prior evaluation settings that isolate individual tasks, restrict tool access, or assume static interaction paradigms. As a result, performance measured under our unified framework better reflects agent behavior in realistic, open-ended user interactions. We further analyze the implications of these properties empirically in Section~\ref{3.3}. 


% \cx{we need to spend way more space in 2.2 as this is our core method, rather than spending too much space in 2.1 about preliminary.}\xiaochuan{check}

% \cx{this subsection focuses too much on the differences with one benchmark, in fact, give the story line, long-context benchmarks may not even be the most relevant related work... single agent tasks are... Instead of focusing too much on differences with one line of work, use this space to state the properties, advnatages and benefits of our work, and in each benefits we can make a contrast with previous tasks. Also maybe we should merge 2.3 and 2.2 as 2.3 is a reflection of 2.2}\xiaochuan{check}

\begin{figure*}[t]
    \centering
    \includegraphics[width=0.99\linewidth]{figs/omni_agenticbench_main.pdf}
    \caption{Relative performance change across domains from the Baseline ($B$) specialized agent setting to the general agent ($G$) setting with unified context and tools. Negative values indicate performance degradation under the General AgentBench.}
    \label{fig:domain_degradation}
\end{figure*}

\subsection{Experimental details}
% \cx{this is kind of experiemntal detail or benchmark constrution and more proper in sec 2?}\xiaochuan{check}
Our evaluation covers a total of ten frontier language models. Among open-source models, we include several high-performing systems such as Qwen3-235B \cite{yang2025qwen3} and DeepSeek-R1 \cite{guo2025deepseek}, as well as more recent models with novel attention mechanisms, including Qwen3-Next \cite{qwen3next} and DeepSeek-v3.2 \cite{liu2025deepseekv3.2}. For proprietary models, we consider both efficiency-oriented variants (e.g., Gemini 2.5 Flash \cite{comanici2025gemini}) and models optimized for complex reasoning (e.g., GPT-5 \cite{openai_gpt5} and Claude Sonnet 4.5 \cite{anthropic2025claude45}). We access these models via Amazon Bedrock \footnote{\url{https://aws.amazon.com/bedrock/pricing/}} and the Hugging Face Inference API .\footnote{\url{https://huggingface.co/docs/inference-providers/en/index}} For all evaluations, we fix the decoding temperature to 0.7 and ensure that each model’s native context length  exceeds the maximum context length required by the benchmark.

% This selection enables a systematic evaluation of emerging attention architectures in realistic agentic task environments, while allowing direct comparison with conventional attention mechanisms. 

% It is therefore necessary to explicitly distinguish long-context model performance in agentic settings from that in traditional long-context benchmarks.

% This makes Omni AgenticBench a rigorous testbed for agentic long-horizon capabilities. 
% Crucially, ``agentic long-context'' differs fundamentally from previous long-context benchmarks, which primarily evaluate single-turn, long-input comprehension. 



% In contrast, Omni AgenticBench grounds long-context evaluation in dynamic scenarios. Agentic tasks inherently generate expansive contexts through multi-turn interactions where the model must maintain a consistent trajectory. This shift moves beyond static assessment to measure a model’s ability to solve complex problems under realistic constraints. We provide further quantitative analysis in Section \ref{3.3}.



\section{Main Results}

In this section, we report overall performance on General AgentBench and compare it against evaluations conducted under prior domain-specific settings to quantify the gap between specialized and general-purpose LLM agents.

\subsection{Result analysis}

Table~\ref{tab:omni_only_domain_avg} summarizes the results across models and domains on General AgentBench. Claude Sonnet 4.5 achieves the strongest overall performance, driven primarily by its tool-use and coding capabilities, while GPT-5 attains the highest scores in the Search and Reason domains, reflecting its strengths in information retrieval and complex reasoning. Among open-source models, DeepSeek-V3.2 outperforms both Gemini variants, demonstrating the significant scaling potential of efficient, sparse-attention architectures. Across models, performance on BrowseComp remains consistently low, indicating that retrieving rare and precise information beyond in-domain training data is still a major bottleneck for current LLM agents.

\begin{figure}[t]
    \centering
    \includegraphics[width=0.95\linewidth]{figs/omni_agenticbench_summary.pdf}
\caption{Performance comparison between specialized-agent and general-agent settings.\textbf{Top}: Absolute performance .\textbf{Bottom}: Relative performance degradation under the general-agent setting.}
    \label{fig:mean_degradation}
\end{figure}

\begin{figure*}[t]
    \centering
    \includegraphics[width=0.92\linewidth]{figs/agentic_scaling_2x4.pdf}
    \caption{\textbf{Test-time scaling behaviors of general LLM agents.} Results are reported for five models across four domains on General AgentBench. \textbf{Top}: Parallel scaling expands the solution space through increased sampling. \textbf{Bottom}: Sequential scaling allocates additional computation via longer interaction histories, yet exhibiting unstable or diminishing returns.}
    \label{fig:tts}
\end{figure*}

We further examine how performance changes when models transition from specialized agents operating under domain-specific contexts to general agents acting within a unified environment with shared toolsets.  Figure~\ref{fig:mean_degradation} summarizes the mean degradation aggregated over all domains, while Figure~\ref{fig:domain_degradation} reports the relative performance change for each agent across domains. Most LLM agents experience substantial degradation in the general-agent setting, with average relative drops ranging from 10\% to 30\%. The magnitude of this degradation varies widely: for example, Gemini 2.5-Pro suffers a drop exceeding 60\% in the Reason domain, falling from top-tier performance in the baseline setting to near-average performance as a general agent. In contrast, Claude Sonnet 4.5 remains notably robust, with only a 0.2\% average degradation. Detailed overall results can be found in Appendix ~\ref{appendix:agentic_benchmark_details}.

\subsection{Cross-domain tool usage}\label{appendix:cross_domain_tool}

Interestingly, several models, including Qwen3-Next, Deepseek-R1, and Claude, exhibit \textbf{performance gains} in the Search domains under the general-agent setting. Trajectory-level analysis shows that these improvements arise from effective \textbf{cross-domain tool usage}, where agents repurpose tools beyond their originally intended domains to support reasoning and information retrieval. 
We take a closer look at these behaviors. Analysis of 189 search task traces from Claude Sonnet 4.5 reveals that $26$\% of tasks ($50/189$) utilized specialized domain tools beyond plain web search. The most frequently used specialized tools include Google Maps APIs ($78$ calls), Paper Search across arXiv, PubMed, and Google Scholar ($60$ calls), and Hugging Face model APIs ($36$ calls). We present a case study demonstrating how domain-specific tools outperform plain web search.

\begin{tcolorbox}[
  colback=white,
  colframe=gray!75,
  title=\textbf{Case Study: Retrieving the Latest HF Model},
  fonttitle=\bfseries\small,
  boxrule=0.6pt,
  left=3pt, right=3pt, top=4pt, bottom=4pt,
  sharp corners,
  breakable
]

{\small \textbf{Task.} Identify the most recent pre-trained language model on HuggingFace suitable for text classification.}

\vspace{5pt}
\begin{tcolorbox}[colback=red!3, colframe=red!15, left=2pt, right=2pt, top=2pt, bottom=2pt, boxrule=0.4pt]
\textbf{\scriptsize Plain Web Search Baseline (6 turns)} \\
\fontsize{6.5pt}{8pt}\selectfont
\texttt{T1: <search> latest Hugging Face pre-trained language\\ 
\phantom{T1: } model text classification 2024 </search>} \\
\texttt{T2: <search> Hugging Face text classification \\ 
\phantom{T2: } models sorted by recent trending 2024 </search>} \\
\texttt{T3: <search> Hugging Face newest text classification \\ 
\phantom{T3: } models January 2025 recently released </search>} \\
\texttt{...} \\
\texttt{T6: <answer> tabularisai/multilingual-sentiment- \\
\phantom{T6: } analysis (fine-tuned DistilBERT) </answer>}
\end{tcolorbox}

\vspace{3pt}

\begin{tcolorbox}[colback=blue!3, colframe=blue!15, left=2pt, right=2pt, top=2pt, bottom=2pt, boxrule=0.4pt]
\textbf{\scriptsize General Agent (3 turns)} \\
\fontsize{6.5pt}{8pt}\selectfont
\texttt{T1: Hugging\_Face\_\_search-models( \\
\phantom{T1: } \{"tags": "text-classification", "limit": $15$\}) \\
\phantom{T1: } $\rightarrow$ [\{"id": "FutureMa/Qwen3-4B-Evasion", \\
\phantom{T1: } "download": $15$\}, \\
\phantom{T1: } \{"id": "BAAI/bge-reranker-v2-m3", \\
\phantom{T1: } "download": $2807076$\}, ...]} \\
\texttt{T2: search\_\_web\_search("ModernBERT HF 2025") \\
\phantom{T2: } $\rightarrow$ "ModernBERT... $8192$-token context, \\
\phantom{T2: } $139$M/$395$M params, trained on $2$T tokens"} \\
\texttt{T3: Hugging\_Face\_\_get-model-info( \\
\phantom{T3: } \{"model\_id": "answerdotai/ModernBERT-base"\}) \\
\phantom{T3: } $\rightarrow$ Full model card and architecture details}
\end{tcolorbox}

\end{tcolorbox}

We observe that the plain search baseline iteratively refines web queries across 6 turns, ultimately finding \texttt{\small tabularisai/multilingual-sentiment-analysis} with only surface-level information (``\texttt{\small fine-tuned DistilBERT}''). In contrast, the General agent system's specialized \texttt{Hugging\_Face\_\_search-models} API directly queries the model hub with structured filters, returning download counts, tags, and model IDs. The subsequent \texttt{Hugging\_Face\_\_get-model-info} call retrieves comprehensive metadata including architecture specifications, training data scale, and official model cards---information unavailable through web search snippets. 

This behavior reflects an agent’s ability to dynamically select and compose tools under minimal domain priors, capturing a more realistic upper bound on general-agent capability and highlighting the importance of evaluation settings that approximate real-world tool availability. 





\section{Test-Time Scaling Evaluation}

% In this section, we present a systematic study of test-time scaling behavior under the omni agentic setting. Our goal is to characterize how general-agent performance evolves with increased test-time computation, and to examine whether performance gains from test-time scaling translate into improvements in real-world agentic interactions.

In this section, we present a systematic study of test-time scaling behavior of general LLM agents. Section~\ref{4.1} introduces the scaling strategies, with results and findings presented in Sections~\ref{4.2} and~\ref{4.3}.


\subsection{Scaling Methodology}\label{4.1}

We investigate test-time scaling through two complementary paradigms: \textbf{Parallel Scaling} and \textbf{Sequential Scaling}, which correspond to distinct axes of computational allocation—breadth of exploration versus depth of exploitation. Compared with more sophisticated test-time scaling techniques such as self-correction\cite{Madaan2023SelfRefine}, beam search\cite{Yao2023ToT}, or MCTS\cite{Zhou2023LATS}, these two strategies are the most commonly adopted and easiest to deploy in real-world agentic systems.

\paragraph{Parallel Scaling.}
In the parallel scaling regime, we independently sample $K$ trajectories for each query. Increasing $K$ expands the reachable action space, thereby increasing the likelihood that the agent explores at least one trajectory containing a correct solution, relative to the single-shot baseline ($K=1$). 

However, in the absence of external oracles or human feedback—particularly in real-world deployments—parallel scaling alone is insufficient: agents must also be capable of evaluating and selecting the best outcome from their own generated trajectories. To assess this ability, we introduce this \textbf{Self-Choice} setting, in which the agent evaluates its parallelly sampled outputs using one of the following strategies:

\textbf{(1) Point-wise choice.}  
The agent independently evaluates each sampled trajectory and assigns a binary judgment. Performance is measured by the alignment between the model’s judgments and oracle labels, averaged over trajectories that are correct under oracle evaluation.

\textbf{(2) Pair-wise choice.}  
The agent compares two sampled trajectories at a time and iteratively promotes the superior one through a bubble-sort-style selection process. After $K-1$ pairwise comparisons, a single trajectory is selected as the final output, and performance is evaluated based on the correctness of this selected trajectory.

Overall, self-choice reflects the practical effectiveness of parallel scaling, while past@$K$ serves as an upper bound that reveals the solution potential.

\paragraph{Sequential Scaling.}
In contrast, sequential scaling increases computational depth by extending the interaction horizon. As the agent engages with the environment, the conversation context progressively grows. When the agent attempts to terminate an episode (e.g., by emitting an End-of-Turn token), we inject an additional round of environment feedback to encourage further reflection on prior reasoning and exploration of alternative solution paths.

To quantify scaling behaviors under both paradigms, we plot task accuracy against the number of independent samples ($K$) for parallel scaling, and against cumulative context length for sequential scaling in Figure~\ref{fig:tts}. Due to the cost of API-based inference, each model is sampled at most four times, and context length is scaled up to 196K tokens. A detailed cost analysis for reproducibility is provided in the Appendix~\ref{appendix:cost}.

\subsection{Sequential Scaling}\label{4.2}

\begin{figure}[h]
    \centering
    \includegraphics[width=0.98\linewidth]{figs/search_heatmap_159_168.pdf}
\caption{\textbf{Instance-level correctness dynamics under sequential scaling.}
We randomly sample 10 instances from the reasoning domain and track their correctness across increasing context lengths. Red indicates incorrect predictions and blue indicates correct ones. Most instances exhibit stagnant or oscillatory behavior, repeating prior successes while failing on unresolved cases, with some fluctuating between correct and incorrect across steps.}
    \label{fig:per_instance}
\end{figure}

Sequential scaling exhibits behavior that departs markedly from our expectations, shown in the bottom of Figure~\ref{fig:tts}. Although several models on certain benchmarks show performance improvements, such as Qwen3-235B on the Search domain and Deepseek-v3.2 on the Reason domain, most show little to no consistent improvement despite allocating additional computational resources through iterative reasoning and reflection. 

We categorize sequential scaling behaviors into two distinct regimes. (1) \textbf{Stagnant fluctuation}: In domains such as reasoning, performance oscillates within a narrow range despite increased computation. This pattern suggests a limited capacity for agents to explore novel solution paths within extended interaction traces, coupled with a diminished ability to maintain coherence—likely due to the agent's finite context processing capacity. (2) \textbf{Saturation and degradation}: In coding domains, models initially benefit from additional reasoning steps; however, beyond an important turning point, performance consistently deteriorates and fails to recover. Figure~\ref{fig:per_instance} shows how instance-level correctness for several data entries changes as the context length increases. We observe that agents either repeatedly succeed on queries they already handle well while failing to progress on unsuccessful cases, or exhibit unstable behavior with accuracy fluctuating between 0 and 1 across interaction steps. These observations further validate our summarization regarding the inherent limitations of sequential scaling.

\begin{figure}[h]
    \centering
    \includegraphics[width=0.98\linewidth]{figs/context_scaling_regions.pdf}
    \caption{Sequential scaling behavior of Gemini~2.5-Flash and Qwen3-235B across domains, with inherent context lengths indicated by the dashed line. Performance scales positively as interaction history approaches and slightly exceeds the inherent context; however, it saturates or degrades once the context extends significantly beyond this threshold. This limit represents the ``context ceiling" of sequential scaling, beyond which further history yields diminishing returns.}
    \label{fig:inherent_context}
\end{figure}


\begin{figure*}[t]
    \centering
    \includegraphics[width=0.98\linewidth]{figs/agentic_scaling_pairwise_facet.pdf}
    \caption{\textbf{Verification gap between generation and self-choice.}
Across four domains, we observe a consistent gap between solution generation and verification: as the number of samples increases, correct solutions appear more frequently in the sampled set, yet models often fail to identify and select them. The dashed and dotted curves represent two self-choice strategies, while the diamond denotes a stronger evaluator, GPT-5.}
    \label{fig:self-choice}
\end{figure*}

To better understand how this turning point occurs, we analyze performance relative to an agent’s inherent context length—defined as the total context naturally accumulated when completing a task without artificial constraints on interaction depth. By integrating data from Qwen3-235B and Gemini 2.5-Flash (as seen in Figure~\ref{fig:tts}) with new experimental results on smaller context windows, we observed a distinct performance ceiling. As shown in Figure~\ref{fig:inherent_context}, both models exhibit an initial upward trend, approaching peak performance as they reach their inherent context limits. However, once the accumulated context passes a specific threshold ( for example, approximately 112K for Qwen3-235B and 96K for Gemini 2.5-Flash in the search domain), performance typically plateaus or begins to degrade. In these instances, additional computation and time yield no further gains. This suggests the existence of a ``\textbf{context ceiling}": a maximum effective context length under sequential scaling, beyond which raw interaction history offers diminishing or zero practical returns. Notably, this ceiling varies across domains, reflecting the unique demands each task places on context utilization and computational efficiency.

Taken together, these results indicate that simply allocating more computation by extending raw interaction histories rarely leads to meaningful performance gains, contradicting previous observations in non-agentic settings \cite{muennighoff2025s1}. Long-horizon agentic tasks expose fundamental challenges in context utilization and reasoning stability, highlighting the need for more effective mechanisms for context management and reasoning control under sequential scaling \cite{zhang2025agentic, claudecontexxt}.

\subsection{Parallel Scaling}\label{4.3}

We observe a monotonic increase in \textbf{pass@K} as the number of sampled trajectories grows (Figure~\ref{fig:tts}, top). Increasing $K$ from 1 to 4 yields an average improvement of roughly 50\%, with DeepSeek-V3.2 exhibiting the largest gains—approaching a twofold improvement in both coding and reasoning domains.

However, pass@K only reflects an idealized upper bound. In practice, the effectiveness of parallel scaling depends on an agent’s ability to evaluate and select among its own sampled trajectories, which is captured by self-choice performance. As shown by the gap between the solid lines (pass@K) and the dashed or dotted lines (self-choice) in Figure~\ref{fig:self-choice}, self-choice performance consistently lags behind the pass@K upper bound regardless of the strategy. In some cases, self-choice performance even degrades as $K$ increases. While the coding domain exhibits the smallest gap between pass@K and self-choice, in most settings self-choice gains saturate quickly and fail to track the continued improvement of pass@K.

To examine whether this gap stems from limited verifier capability, we replace the model’s internal self-judgment with an external verifier, GPT-5, and perform point-wise evaluation of sampled trajectories from all models. As shown by the diamond line in Figure~\ref{fig:self-choice}, GPT-5 generally underperforms models’ own self-judgment. Notably, even at $K=1$, GPT-5 occasionally misclassifies correct trajectories as incorrect, introducing a non-trivial gap between pass@1 and GPT-5–based verification. We hypothesize that this effect arises from \textbf{solution familiarity}: models are better at evaluating their own generations, which align closely with their internal reasoning patterns, whereas external verifiers may struggle to accurately assess unfamiliar execution traces.

Overall, these results reveal a fundamental gap between models’ generation capacity and their ability to reliably select correct solutions, which ultimately limits the practical utility of parallel scaling. 

% Bridging this gap requires improving agents’ self-evaluation robustness and reliability, rather than solely expanding the generation space.




% \cx{I think it is clear that now we should treat the pairwise choice as the parallel scaling, as it is the actual scaling method, then the pass at K is the upper bound to show the verification gap... hopefully we have all the results to change the plots, and the writing changes can be done in the next pass. Right now the oracle pass at K upper bound is mixed with the actual sequence scaling, making it quite confusing}\xiaochuan{check}

% \cx{then you can move the pairwise self choice descriptions as part of the previou subsection as scaling methods (and it is realy a scaling method now)}\xiaochuan{check}



% \cx{will be great if we can show the inherit reasoning context length of at least one model's distributions on all the testing user requests, to show this important message, likely here in this paper...}\xiaochuan{check}



% \section{Attention Analysis}\label{section5}

\subsection{Comparison between Linear and Full Attention}

Across both parallel and sequential test-time scaling, Qwen3-Next shows weaker performance gains than full-attention models. To investigate this gap, we analyze the impact of its hybrid attention design. We compare it with Qwen3-235B-A22B, a full-attention baseline, by measuring average attention distance in the final reasoning step. Specifically, for every full-attention head, we record the top-$k$ attended tokens and compute their token-distance relative to the beginning of the final output span. For Gated DeltaNet, which does not explicitly expose standard attention weights, we adopt a mathematically equivalent procedure to recover the effective top-$k$ contributing tokens and compute the corresponding distances (details provided in Appendix~X).

Figure~\ref{fig:attention_analysis} visualizes the resulting heatmaps. We find that full attention consistently exhibits larger average attention distances than Gated DeltaNet. This is evident both in the comparison between Qwen3-235B and Qwen3-Next, as well as within Qwen3-Next itself when contrasting its Gated Attention and DeltaNet components. These results indicate that full attention is more effective at capturing long-range dependencies, which likely contributes to its superior scaling consistency under long-context agentic settings


% \cx{all sec 5 are interesting, but too verbose, can be much more concise and reduce the space into half? I really want to inherit reasoning token budget study as it illustates a new and clear picture of what seq scaling does not beyond certain token lens. then if we have space we can move some case studies to main paper}\xiaochuan{check. Shortened the three subsections.}

\begin{figure}[t]
    \centering

    \begin{subfigure}[t]{0.48\linewidth}
        \centering
        \includegraphics[width=\linewidth]{figs/attention_mean_distance_qwen3_235b.pdf}
        \caption{Qwen3-235B (full attention)}
        \label{fig:attn_qwen3_235b}
    \end{subfigure}
    \hfill
    \begin{subfigure}[t]{0.48\linewidth}
        \centering
        \includegraphics[width=\linewidth]{figs/attention_mean_distance_qwen3_next.pdf}
        \caption{Qwen3-Next (hybrid attention)}
        \label{fig:attn_qwen3_next}
    \end{subfigure}

    \caption{\textbf{Mean top-$k$ attention distance across layers and heads.}
    We report the average distance between the top-$k$ attended tokens and the beginning of the model’s final output span.
    \textbf{Left}: Qwen3-235B with full attention.
    \textbf{Right}: Qwen3-Next with hybrid Gated Attention and Gated DeltaNet.}
    \label{fig:attention_analysis}
\end{figure}
\section{Attention Behavior Analysis}\label{section5}

\begin{figure*}[h]
    \centering

    \begin{subfigure}{0.9\linewidth}
        \centering
        \includegraphics[width=\linewidth]{figs/attention_analysis_final_qwen3_235B.pdf}
        % \caption{Qwen3-235B}
        \label{fig:attn_qwen3_235b}
    \end{subfigure}

    \vspace{6pt}

    \begin{subfigure}{0.9\linewidth}
        \centering
        \includegraphics[width=\linewidth]{figs/attention_analysis_final_qwen3_next.pdf}
        % \caption{Qwen3-Next}
        \label{fig:attn_qwen3_next}
    \end{subfigure}

    \caption{Comparison of attention behaviors under the General AgentBench setting. Top: Qwen3-235B (full attention). Bottom: Qwen3-Next (hybrid linear attention).}
    \label{fig:attention_analysis}
\end{figure*}


In sequential test-time scaling, we observe that Qwen3-Next demonstrates weaker scaling potential compared to other models, particularly its full-attention counterpart, Qwen3-235B-A22B. Although these models differ in training data composition and parameterization, we analyze this discrepancy primarily from the perspective of the attention mechanism—the most salient architectural difference between them. We first introduce our analysis methodology, followed by empirical results and key findings.
 

\subsection{Extracting Top-K Attention Tokens for Reasoning Behavior}

Our attention analysis is conducted on inference trajectories collected from General AgentBench. For each benchmark domain, we randomly sample 25 trajectories generated by the model under analysis. Because attention patterns depend strongly on the specific text region being examined, we follow the framework of \citet{jin2025beneficial} to extract reasoning-behavior sentences from each trajectory. These sentences correspond to critical decision-making steps, where the model actively reasons over accumulated context. For each reasoning sentence, we iterate over all tokens within the sentence. For every token, we compute and store its attention distribution over the preceding context. We then average the attention scores across tokens in the reasoning sentence, select the top-$K$ attended tokens ($K = 128$), and record their token indices. This procedure is applied to all layers and attention heads.

We evaluate attention behavior using two metrics:

\paragraph{Mean Attention Distance.} For each of the top-$K$ attended tokens, we compute its token distance from the reasoning sentence and weight it by the corresponding attention score. This metric measures how far back the model attends when making a key decision, effectively quantifying the model’s effective contextual view.

\paragraph{Top-K Overlap.}
We measure the overlap of top-$K$ attended tokens through intra-layer overlap (across heads within the same layer) and inter-layer overlap ( across different layers). Higher overlap indicates that different heads or layers attend to similar contextual tokens, suggesting reduced functional differentiation.

For Gated DeltaNet, which does not explicitly expose standard attention weights, we adopt a mathematically equivalent reconstruction procedure to recover the effective top-$K$ contributing tokens and compute the corresponding distance statistics.

\subsection{Results and findings}

In the leftmost panel of Figure~\ref{fig:attention_analysis}, we visualize the mean attention distance for each head across layers. Full-attention models exhibit consistently larger mean distances, with only minor exceptions in early layers and around layer 70. In most layers, the majority of heads attend to long-range context, while a small subset focuses on local patterns. This behavior is also observed in the gated full-attention layers of Qwen3-Next (appearing every four layers). These results suggest that full attention maintains a broader effective contextual view than linear attention, consistent with the convolution-like receptive field constraints imposed by DeltaNet-style linear attention \cite{yang2024gated}.

The two rightmost panels present the top-$K$ token overlap statistics.

\paragraph{Intra-layer overlap.}
Full attention exhibits a characteristic V-shaped curve: middle-layer heads attend to more diverse patterns, while later layers converge toward similar tokens, reflecting increased certainty near the final decision stage. In contrast, linear attention lacks a clear structural pattern and shows higher average intra-layer overlap, suggesting reduced head specialization.

\paragraph{Inter-layer overlap.}
Full attention displays a gradual ``low-to-high'' trend across depth, indicating that adjacent layers share similar functional roles, while functional divergence accumulates progressively with depth. In linear-attention models, DeltaNet layers show very low overlap with gated full-attention layers. Although DeltaNet layers exhibit high overlap across distant layers (indicating homogeneous behavior), the full-attention layers preserve their characteristic inter-layer structure.

Overall, our findings indicate that linear attention demonstrates weaker functional differentiation across heads and layers, along with reduced long-context utilization in agentic reasoning tasks, compared to full-attention mechanisms.

\section{Related Work}

Prior work on LLM agents spans three closely related directions: benchmarking agentic abilities, developing agent models and frameworks, and exploring effective scaling methods. Most agentic benchmarks evaluate specialized agents in domain-specific settings, where tasks, interaction protocols, and tool access are tailored to specific skills. Representative ones cover software engineering (e.g., SWE-bench \cite{jimenez2023swebench} and its variants \cite{Aleithan2024SWEBenchPlus,Zhang2025SWEBenchGoesLive,yang2024sweagent}), web navigation and interaction (e.g., WebShopcite \cite{Yao2022WebShop}, Mind2Web \cite{Deng2023Mind2Web}, WebArena \cite{zhou2023webarena}, WebVoyager \cite{he2024webvoyager}), and tool use with curated APIs (e.g., ToolLLM \cite{Qin2023ToolLLM}, API-Bank \cite{Li2023APIBank}, BFCL \cite{patil2025bfcl}, StableToolBench \cite{Guo2024StableToolBench}). Broader suites such as $\tau$-bench \cite{Yao2024TauBench}, AgentBench\cite{barres2025tau2}, and GAIA \cite{Mialon2023GAIA} incorporate multi-turn interaction across multiple task categories , while OSWorld \cite{xie2024osworld}, OSWorld-G \cite{xie2025jedi}, and AndroidWorld \cite{rawles2024androidworld} evaluate agents operating in real desktop and mobile systems.

In parallel, substantial effort has focused on building general-purpose agents capable of planning, acting, and invoking tools across heterogeneous tasks. Early methods such as ReAct \cite{yao2022react}, Reflexion \cite{shinn2023reflexion}, Toolformer \cite{schick2023toolformer}, and HuggingGPT \cite{Shen2023HuggingGPT} introduce structured reasoning–action trajectories, reflection, and learned tool invocation. More recent work emphasizes scalable agent frameworks and deployment platforms supporting multi-agent coordination and rich tool ecosystems, including OpenAgents \cite{Xie2023OpenAgents}, AgentVerse \cite{chen2023agentverse}, AgentScope \cite{gao2024agentscope}, and OpenCUA \cite{wang2025opencua}. Industry systems—such as agents built on Claude \cite{claudeagent}, Microsoft \cite{microsoftagent}, OpenAI \cite{openaiagent}, Qwen \cite{Qwenagent}, Kimi \cite{moonshotai2026kimi-k25}, and Gemini \cite{gemini}—further demonstrate the practical importance of general agents in real-world deployments.

A complementary line of research studies test-time scaling, allocating additional inference-time computation to improve agent performance. Chain-of-thought prompting \cite{Wei2022CoT} and self-consistency \cite{Wang2022SelfConsistency} show that deeper reasoning and multi-sample decoding can substantially boost accuracy. More explicit approaches perform search or refinement over solution paths, including Tree-of-Thoughts \cite{Yao2023ToT}, MCTS-based agent planning such as LATS \cite{Zhou2023LATS}, and iterative refinement methods like Reflexion \cite{shinn2023reflexion} and Self-Refine \cite{Madaan2023SelfRefine}. More recently, ``internal scaling’’ trains models to autonomously decide how much inference computation to allocate and when to terminate reasoning, shifting control from external orchestration toward model-internal deliberation \cite{guo2025deepseek, openai2024learning-to-reason}. Verifier-based inference further augments sampling or search with learned ranking or rejection, with process-level supervision improving reliability in mathematical reasoning \cite{ Lightman2023VerifyStepByStep, li2024montessori, hosseini2024v, wang2024math}. 

\section{Conclusions}
We present General AgentBench, a unified benchmark for evaluating LLM agents under realistic, multi-domain interactions where agents must infer intent, select tools from a shared pool, and act end-to-end. Across ten leading models, we find a substantial robustness gap when moving from domain-specific to general-agent evaluation. Our test-time scaling analysis reveals two fundamental limits: sequential scaling is bounded by an context ceiling, beyond which longer interactions become unstable, and parallel scaling delivers limited practical gains due to a persistent verification gap between generation and self-choice. We hope General AgentBench enables realistic assessment and guides progress toward robust, scalable general agents.



% \section*{Software and Data}

% If a paper is accepted, we strongly encourage the publication of software and
% data with the camera-ready version of the paper whenever appropriate. This can
% be done by including a URL in the camera-ready copy. However, \textbf{do not}
% include URLs that reveal your institution or identity in your submission for
% review. Instead, provide an anonymous URL or upload the material as
% ``Supplementary Material'' into the OpenReview reviewing system. Note that
% reviewers are not required to look at this material when writing their review.

% Acknowledgements should only appear in the accepted version.
% \section*{Acknowledgements}

% \textbf{Do not} include acknowledgements in the initial version of the paper
% submitted for blind review.

% If a paper is accepted, the final camera-ready version can (and usually should)
% include acknowledgements.  Such acknowledgements should be placed at the end of
% the section, in an unnumbered section that does not count towards the paper
% page limit. Typically, this will include thanks to reviewers who gave useful
% comments, to colleagues who contributed to the ideas, and to funding agencies
% and corporate sponsors that provided financial support.

\section*{Impact Statement}

This paper introduces General AgentBench, a benchmark for evaluating large language model agents as general-purpose systems under unified and realistic interaction settings. By exposing agents to diverse tasks and tools within a single framework, it enables more consistent and transparent assessment of agents’ abilities to interpret open-ended requests, select appropriate tools, and scale inference-time computation across domains. We expect this benchmark to support the development and diagnosis of more robust general-purpose agents. We also acknowledge potential risks. Unified evaluation and test-time scaling may favor computationally intensive or proprietary models, and unreliable self-choice under parallel scaling may lead to unstable or misleading agent behavior if applied naively in deployment. As general-purpose agents are increasingly integrated into real-world applications, failures in long-horizon reasoning, tool use, or self-verification may affect reliability and user trust. We emphasize that General AgentBench is intended for research and evaluation purposes. The community shares a responsibility to interpret results carefully and to pair scaling-based improvements with stronger verification, transparency, and safety considerations.
    
\bibliography{icml_references}
\bibliographystyle{icml2026}

%%%%%%%%%%%%%%%%%%%%%%%%%%%%%%%%%%%%%%%%%%%%%%%%%%%%%%%%%%%%%%%%%%%%%%%%%%%%%%%
%%%%%%%%%%%%%%%%%%%%%%%%%%%%%%%%%%%%%%%%%%%%%%%%%%%%%%%%%%%%%%%%%%%%%%%%%%%%%%%
% APPENDIX
%%%%%%%%%%%%%%%%%%%%%%%%%%%%%%%%%%%%%%%%%%%%%%%%%%%%%%%%%%%%%%%%%%%%%%%%%%%%%%%
%%%%%%%%%%%%%%%%%%%%%%%%%%%%%%%%%%%%%%%%%%%%%%%%%%%%%%%%%%%%%%%%%%%%%%%%%%%%%%%
\newpage
\appendix
\onecolumn
% \section{You \emph{can} have an appendix here.}

% You can have as much text here as you want. The main body must be at most $8$
% pages long. For the final version, one more page can be added. If you want, you
% can use an appendix like this one.

% The $\mathtt{\backslash onecolumn}$ command above can be kept in place if you
% prefer a one-column appendix, or can be removed if you prefer a two-column
% appendix.  Apart from this possible change, the style (font size, spacing,
% margins, page numbering, etc.) should be kept the same as the main body.
%%%%%%%%%%%%%%%%%%%%%%%%%%%%%%%%%%%%%%%%%%%%%%%%%%%%%%%%%%%%%%%%%%%%%%%%%%%%%%%
%%%%%%%%%%%%%%%%%%%%%%%%%%%%%%%%%%%%%%%%%%%%%%%%%%%%%%%%%%%%%%%%%%%%%%%%%%%%%%%

\section{Additional Experimental Results}

\subsection{Agentic Benchmarks}

\paragraph{Coding} We include tasks from SWE-Bench Verified\cite{openai_swebench_verified_2024} and Terminal Bench for the code domain. SWE-Bench Verified consists of real-world GitHub issues that require the model to analyze and propose concrete bug fixes, with evaluation performed through automated test pass rates. Terminal Bench assesses a model’s ability to solve problems within a terminal environment, requiring not only command-line common sense but also the ability to plan and reason over long user instructions. It examinzes the score by checking whether the model achieves the expected final state. 


\paragraph{Reason} For this domain, we adopt MathHay\cite{wang2024mathhay} as our data source. MathHay collects mathematical information grounded in real web documents. For each group of related documents, specific pieces of information are extracted and linked to form a query. The benchmark then constructs a long-context haystack by inserting these relevant documents into noisy text placed at the beginning, middle, or end of the sequence. This design provides a suitable testbed for evaluating the model's long‑context reasoning ability. One thing to note is that no raw tools are provided in this benchmark.

Most reasoning benchmarks provide all necessary information directly within the query and ask the model to perform complex mathematical or scientific derivations. As a result, they naturally lack an interactive environment that can deliver feedback, making them unsuitable for multi-turn evaluation. However, we can simulate multi-turn interaction by explicitly prompting the model to refine its own answers over additional turns, effectively creating self-reflective reasoning cycles. A key requirement for such evaluation is that the query context must be sufficiently long to allow meaningful iterative improvement. 

\paragraph{Search} For the search domain, we include tasks from BrowseComp\cite{wei2025browsecomp} and WebVoyager\cite{he2024webvoyager}, aiming to evaluate a model’s ability to locate accurate information within long, evolving contexts. In contrast to Needle-in-a-Haystack or purely textual retrieval tasks, agentic search requires the model to reason about what information is needed, determine whether the current context is sufficient to answer the query, and decide when additional search steps are necessary. Mind2Web and WebVoyager focus on everyday web-browsing tasks—such as shopping, navigation, and entertainment—and rely on external language models or task-specific agents to assess model performance. BrowseComp, in contrast, presents challenging, hard-to-find information queries that require multi-step investigation across the web. The model’s final answers are compared against expert-curated gold references to produce the evaluation score.

\paragraph{Tool-use} For the tool-use domain, we include tasks from Tau2-Bench\cite{barres2025tau2} and MCP-Bench\cite{wang2025mcpbench}, both of which provide extensive tool suites for evaluating a model’s ability to understand, select, and invoke the appropriate tool to solve a problem. Tau2-Bench focuses on customer-service scenarios, where the model acts as a customer-support agent responding to simulated user queries using a set of synthetic tools tailored to the scenario. MCP-Bench, in contrast, offers a collection of real-world tools built on the Model Context Protocol (MCP), requiring models to perform dense tool calling and coordinate across multiple tools to complete complex tasks. 

Both benchmarks require multi-turn interaction with a simulated environment, and the complexity and richness of the task specifications make them nearly unsolvable within a short context window.


While the coding and search domains can be viewed as specific instances of tool use—since they also involve code-execution tools and search APIs—the tool-use domain in General AgentBench significantly broadens the definition of “tools.” It requires models to precisely control tool parameters, reason about which tool to invoke and when, and plan multi-step interactions based on previous own generation and external feedback.

\subsection{Detailed Results Comparison}~\label{appendix:agentic_benchmark_details}

Table~\ref{tab:baseline_vs_general} reports the detailed performance of all evaluated models on both the original domain-specific benchmarks and our General AgentBench setting. These results serve as the data source for Figure~\ref{fig:domain_degradation} and Figure~\ref{fig:mean_degradation}.

\definecolor{neg}{HTML}{C00000} % Red for performance drop
\newcommand{\loss}[1]{\color{neg}#1}

\begin{table*}[t]
\centering
\small
\caption{Main Results on \textbf{General AgentBench}. We report performance in the Baseline ($B$) and General ($G$) settings. $\Delta\%$ represents the relative change. The overall average scores ($\text{Avg}_B, \text{Avg}_G$) and the mean relative degradation are averaged across four domains instead of each benchmark. Bold text indicates the best performance, while underlining denotes the second best; red highlights performance degradation.}
\label{tab:baseline_vs_general}
\setlength{\tabcolsep}{3.2pt}
\begin{tabular}{@{} l *{4}{ccr} cc r @{}}
\toprule
\multirow{2.5}{*}{\textbf{Models}} &
\multicolumn{3}{c}{\textbf{Search}} &
\multicolumn{3}{c}{\textbf{Code}} &
\multicolumn{3}{c}{\textbf{Reason}} &
\multicolumn{3}{c}{\textbf{Tool-Call}} &
\multicolumn{2}{c}{\textbf{Overall Avg.}} &
\multirow{2.5}{*}{\textbf{Avg. $\Delta\%$}} \\
\cmidrule(lr){2-4} \cmidrule(lr){5-7} \cmidrule(lr){8-10} \cmidrule(lr){11-13} \cmidrule(lr){14-15}
& $B$ & $G$ & $\Delta\%$
& $B$ & $G$ & $\Delta\%$
& $B$ & $G$ & $\Delta\%$
& $B$ & $G$ & $\Delta\%$
& $\text{Avg}_B$ & $\text{Avg}_G$ & \\
\midrule

\textit{Open-Source} & & & & & & & & & & & & & & & \\

GPT-OSS-120B
& 17.6 & 12.1 & \loss{-31.3}
& 16.9 & 8.5 & \loss{-49.7}
& 46.7 & 38.7 & \loss{-17.1}
& 65.2 & 45.0 & \loss{-31.0}
& 36.6 & 26.1 & \loss{-28.7} \\

Qwen3-235B-A22B
& 21.7 & 16.4 & \loss{-24.4}
& 36.2 & 22.5 & \loss{-37.8}
& 45.3 & 32.0 & \loss{-29.4}
& 62.6 & 52.5 & \loss{-16.1}
& 41.5 & 30.9 & \loss{-25.5} \\

Qwen3-Next
& 16.1 & 19.1 & +18.6
& 22.3 & 12.3 & \loss{-44.8}
& 44.0 & 42.0 & \loss{-4.5}
& 63.3 & 56.9 & \loss{-10.1}
& 36.4 & 32.6   & \loss{-10.4} \\

DeepSeek-V3.2
& \underline{34.6} & 28.6 & \loss{-17.3}
& 32.3 & 25.9 & \loss{-19.8}
& 34.7 & 33.3 & \loss{-4.0}
& 61.8 & \underline{60.1} & \loss{-2.8}
& 40.9 & 37.0 & \loss{-9.5} \\

DeepSeek-R1
& 16.6 & 21.2 & +27.7
& 23.1 & 10.8 & \loss{-52.2}
& 47.3 & \underline{46.7} & \loss{-1.3}
& 44.7 & 40.1 & \loss{-10.3}
& 32.9 & 29.7 & \loss{-9.8} \\

\addlinespace[0.5em]
\textit{Proprietary} & & & & & & & & & & & & & & & \\

Gemini 2.5-Flash
& 21.1 & 15.4 & \loss{-27.0}
& 30.0 & 17.7 & \loss{-41.0}
& 60.0 & 36.0 & \loss{-40.0}
& 66.0 & 52.7 & \loss{-20.2}
& 44.3 & 30.5 & \loss{-31.2} \\

Gemini 2.5-Pro
& 23.4 & 21.7 & \loss{-7.3}
& 26.9 & 26.9 & 0.0
& \underline{61.3} & 24.0 & \loss{-60.8}
& 66.1 & 56.8 & \loss{-14.1}
& 44.4 & 32.4 & \loss{-27.2} \\

Claude Haiku 4.5
& 23.6 & 28.0 & +18.6
& 41.5 & 36.9 & \loss{-11.1}
& 54.7 & 34.7 & \loss{-36.6}
& 64.5 & 56.7 & \loss{-12.1}
& \underline{46.1} & 39.1 & \loss{-15.2} \\

Claude Sonnet 4.5
& 26.1 & \underline{34.7} & +33.0
& \bfseries 49.2 & \bfseries 48.5 & \loss{-1.4}
& 32.0 & 36.0 & +12.5
& \underline{73.0} & \bfseries 60.7 & \loss{-16.8}
& 45.1 & \underline{45.0} & \loss{-0.2} \\

GPT-5
& \bfseries  55.8  & \bfseries 39.1 & \loss{-29.9}
&  \underline{45.4} & \underline{39.3} & \loss{-13.4}
& \bfseries 64.0 & \bfseries 64.0 & 0.0
& \bfseries 78.3 & 45.8 & \loss{-41.5}
& \bfseries 60.9 & \bfseries 47.1 & \loss{-22.7} \\

\bottomrule
\end{tabular}
\end{table*}
 




\begin{figure}[h]
        \centering
        \includegraphics[width=0.8\linewidth]{figs/task_correlation.pdf}
        \caption{\textbf{Pairwise correlation between static long-context benchmarks and agentic domains.} “All” denotes the average performance across the search, code, reasoning, and tool-use domains.}
        \label{fig:corr}
\end{figure}

\begin{figure}[h]
        \centering
        \includegraphics[width=0.8\linewidth]{figs/model_pair_wise_consistency.pdf}
        \caption{\textbf{Pairwise correlation between models.}}
        \label{fig:corr_models}
\end{figure}

\subsection{Comparison with related long-context benchmarks} \label{appendix:long_compare}

\begin{table}[ht]
\centering
\small
\caption{Performance on Traditional Long-Context Benchmarks. We report scores across three established benchmarks focusing on retrieval and single-turn comprehension. Bold indicates the best performance in each category.}
% \caption{Performance on Traditional Long-Context Benchmarks. We report scores across three established benchmarks focusing on retrieval and single-turn comprehension. Bold indicates the best performance in each category.}
\label{tab:long_context_baselines}
\setlength{\tabcolsep}{10pt}
\begin{tabular}{@{} l ccc @{}}
\toprule
\textbf{Models} & \textbf{LongBench} & \textbf{HELMET} & \textbf{MRCR} \\
\midrule
\textit{Open-Source} & & & \\
GPT-OSS-120B      & 47.8 & 12.9 & 32.8 \\
Qwen3-235B-A22B   & 58.3 & 40.8 & 40.6 \\
Qwen3-Next        & 53.1 & 26.3 & 27.9 \\
DeepSeek-V3.2     & 50.3 & 48.0 & 33.2 \\
DeepSeek-R1       & 58.3 & 36.9 & 39.2 \\
\addlinespace[0.5em]
\textit{Frontier} & & & \\
Gemini 2.5-Flash  & 62.1 & 27.8 & 67.4 \\
Gemini 2.5-Pro    & 63.3 & \bfseries 63.1 & \bfseries 80.0 \\
Claude Haiku 4.5  & 55.3 & 50.4 & 33.3 \\
Claude Sonnet 4.5 & 61.8 & 49.2 & 35.5 \\
GPT-5             & \bfseries 64.6 & 51.6 & 79.1 \\
\bottomrule
\end{tabular}
\end{table}

With unified toolsets alone already approaching 64K tokens, the addition of user queries and multi-turn interaction histories can easily push the total context length to nearly 128K tokens. As a result, long-context processing naturally emerges as a core capability required for general agents. However, existing long-context benchmarks differ fundamentally from agentic long-context scenarios. As summarized in Table~\ref{tab:taxonomy}, we identify two key dimensions along which prior benchmarks diverge from agentic settings.

(1) \textbf{Context composition.} Most existing long-context benchmarks are dominated by long-document question answering (QA), where the interaction paradigm remains static and single-turn. In contrast, agentic contexts are inherently heterogeneous: beyond long documents, they include environment feedback (e.g., tool execution results) and the model’s own prior decisions accumulated through multi-turn interactions.

(2) \textbf{Long-output reasoning.} Other established benchmark categories, such as many-shot in-context learning and summarization, involve long inputs but require only relatively short outputs. Agentic tasks, however, demand sustained reasoning over extended contexts, including plan generation, iterative reflection, and explicit tool-call specifications, often resulting in long and structured outputs. While retrieval-based tasks and recent citation-grounded generation benchmarks partially resemble agentic scenarios, they still lack the multi-turn, interactive dynamics essential to true agentic settings.

These fundamental differences suggest that performance measured on prior non-agentic long-context benchmarks does not directly reflect model behavior under agentic interaction. 

\begin{table*}[ht]
\centering
\small
\caption{Taxonomy of Existing Long-Context Benchmarks. Unlike static evaluation paradigms, Agentic LongBench introduces multi-turn dynamics and sequential reasoning over expansive contexts.}
\label{tab:taxonomy}
\begin{tabularx}{\textwidth}{l l X}
\toprule
\textbf{Task Category} & \textbf{Sub-category} & \textbf{Representative Benchmarks / Papers} \\ 
\midrule
\textbf{Long Document QA} & Single/Multi-Doc Understanding & NarrativeQA \cite{kovcisky2018narrativeqa}, 2WikiMultihopQA \cite{ho-etal-2020-constructing}, Loong \cite{wang2024leave} \\
\cmidrule{2-3}
& Long Dialogue Understanding & MeetingBank \cite{hu2023meetingbank}, CharacterChat \cite{tu2023characterchat} \\
\cmidrule{2-3}
& Code Understanding & RepoBench \cite{liu2023repobench}, CrossCodeEval \cite{ding2023crosscodeeval} \\
\cmidrule{2-3}
& Structured Data Understanding & TabFact \cite{chen2019tabfact}, WikiTableQuestions \cite{kweon2023open}, LongBench \cite{bai2024longbench} (L-Data)\\
\midrule
\textbf{Summarization} & Global Information Aggregation & GovReport \cite{huang2021efficient}, Multi-News \cite{fabbri2019multi}, ZeroSCROLLS \cite{shaham2023zeroscrolls} \\
\midrule
\textbf{Many-Shot} & In-Context Learning & MIR-Bench \cite{yan2025mir}, Many-Shot ICL \cite{agarwal2024many} \\
\midrule
\textbf{Retrieval} & Key-Value Retrieval & Needle In A Haystack \cite{kamradt_needlehaystack_2023}, RULER \cite{hsieh2024ruler}, MRCR \cite{openai_mrcr_2025}, BABILong \citeyear{kuratov2024babilong} \\
\cmidrule{2-3}
& Retrieval-Augmented Gen (RAG) & RAGBench \cite{friel2024ragbench}, RGB \cite{chen2024benchmarking}, RECALL \cite{liu2023recall} \\
\midrule
\textbf{Reranking} & Passage Reranking & BEIR \cite{thakur2021beir}, MTEB \cite{muennighoff2023mteb} (Reranking Selection) \\
\midrule
\textbf{Citation Gen} & Generation with Citations & ALCE \cite{gao2023enabling} (ASQA/ELI5), LongCite \cite{zhang2025longcite} \\
\midrule
\rowcolor[gray]{0.9} \textbf{Agentic } & \textbf{Multi-turn Reasoning} & SWE-Bench \cite{jimenez2023swebench}, BrowseComp \cite{wei2025browsecomp}, OSWorld \cite{xie2024osworld}, \textbf{General AgentBench (This Work)} \\
\bottomrule
\end{tabularx}
\end{table*}



\subsection{Transferability of Long-Context Abilities from Static to Agentic Benchmarks}\label{appendix:long}

Long-context processing is a fundamental requirement for general agents, as unified toolsets and accumulated multi-turn interactions can rapidly extend the effective context length during real-world use. However, it remains unclear whether performance measured on existing \emph{static}, single-turn long-context benchmarks meaningfully transfers to \emph{agentic} settings, where context evolves dynamically through interaction and decisions must be made sequentially. To study this question, we evaluate models on several representative static long-context benchmarks and examine how their performance correlates with results on General AgenticBench. We choose LongBench, HELMET, and MRCR here.

\paragraph{LongBench v2}
LongBench v2 is an updated and expanded version of LongBench, designed to evaluate language models under diverse long-context understanding tasks across multiple domains. It covers a broad set of settings, including long-document question answering, multi-document reasoning, summarization, code understanding, and many-shot in-context learning. Compared to the original LongBench, v2 increases both context length and task diversity, and introduces harder examples that stress retrieval accuracy and reasoning robustness under long inputs. Despite its breadth, LongBench v2 remains fundamentally \emph{single-turn} and \emph{static} in interaction structure. All necessary information is provided upfront, and models are evaluated on their ability to extract, aggregate, or reason over relevant spans within a fixed context window. Outputs are typically short and final, without iterative refinement or environment feedback. As a result, LongBench v2 primarily measures long-context comprehension and retrieval, rather than the dynamic decision-making, planning, and self-conditioning behaviors required in agentic multi-turn settings.

\paragraph{HELMET}
HELMET is a holistic benchmark for evaluating long-context language models across a curated set of real-world and synthetic tasks. It emphasizes robustness, faithfulness, and information utilization under long inputs, and includes task categories such as long-document QA, summarization, citation-grounded generation, and structured information extraction. HELMET is particularly focused on evaluating whether models can correctly attribute claims to source documents and avoid hallucinations when operating over extended contexts.

\paragraph{MRCR}
MRCR focuses on evaluating a model’s ability to perform multi-round coreference and entity tracking under long contexts. The benchmark constructs documents containing repeated, interleaved references to entities across long spans, requiring the model to resolve pronouns, aliases, and implicit references over multiple turns or segments. Tasks are designed to test memory persistence and consistency, particularly in scenarios where earlier mentions are far removed from later queries.



Across ten models, we compute Pearson correlations over pairwise absolute performance differences between static long-context benchmarks and the four agentic domains. As shown in Figure~\ref{fig:corr}, static benchmarks exhibit consistently weak correlation with agentic performance overall, indicating limited transferability from static long-context ability to agentic long-context reasoning. A moderate correlation is observed between MRCR and the reasoning domain, which is expected: reasoning tasks primarily involve extracting and computing over long documents without tool interaction, closely aligning with the characteristics of MRCR. Other agentic domains show substantially lower alignment. In particular, coding and tool-use exhibit minimal correlation with static benchmarks, suggesting that agentic performance depends not only on long-context comprehension, but also on dynamic decision-making and precise execution. These results highlight a fundamental gap between static, single-turn long-context evaluation and the requirements of realistic agentic settings. We further provide correlations across different models on the General AgenticBench as shown in Figure~\ref{fig:corr_models}.



\section{Estimated Cost}\label{appendix:cost}

\subsection{Pricing and accounting.}
We report an estimated API budget for reproducing our evaluation under three settings:
(i) \textbf{general (default context)} evaluation of each model on each domain once,
(ii) \textbf{parallel scaling} that samples multiple independent trajectories per query, and
(iii) \textbf{sequential scaling} that extends the interaction horizon over multiple steps.

All prices are normalized as USD per 1M tokens and are taken from the corresponding model providers at the time of running the experiments. Due to uncertainty about the model provider's caching mechanism and evaluation intervals, we will use only the input unit price in our estimates, even if the provider supplies a cache unit price. 

We compute cost by aggregating token usage from execution logs, using the provider-reported token accounting:
\textbf{input tokens} (prompt + tool outputs + intermediate messages fed back to the model),
and \textbf{output tokens} (model-generated tokens).
For each model, the total cost is estimated as:
\[
\text{Cost} =
p_{\text{in}}\cdot \tfrac{T_{\text{in}}}{10^6}
+ p_{\text{out}}\cdot \tfrac{T_{\text{out}}}{10^6},
\]
where $p_{\text{in}}, p_{\text{out}}$ denote the unit prices, and
$T_{\text{in}}, T_{\text{out}}$ are the measured token counts.


\subsection{Model cost}

Table~\ref{tab:lite-version-cost} lists the unit API prices used in our cost estimation. Prices for input, cached input, and output are reported in USD per 1M tokens. Tables~\ref{tab:parallel-scaling-cost}--\ref{tab:general-setting-cost} summarize the aggregated evaluation cost per model under parallel scaling, sequential scaling, and the general (default context) setting, respectively.
Each entry corresponds to running the full benchmark split of that dataset/domain under the specified protocol and then summing costs across all queries.

\begin{table}[h]
\centering
\small
\begin{tabular}{lccc}
\toprule
\textbf{Model} & \textbf{Input} & \textbf{Cached Input} & \textbf{Output} \\
\midrule
Gemini-2.5-Flash      & 0.30 & 0.03  & 2.50 \\
Gemini-2.5-Pro        & 1.25 & 0.125 & 10.00 \\
GPT-5                 & 1.25 & 0.125 & 10.00 \\
Claude-Haiku 4.5      & 1.00   & 0.50    & 5.00 \\
Claude-Sonnet 4.5     & 3.00 & 3.75  & 15.00 \\
\midrule
gpt-oss-120B           & 0.15 & 0.075    & 0.60 \\
Deepseek-R1           & 1.35 & --    & 5.40 \\
Deepseek-V3.2         & 0.28 & --    & 0.42 \\
Qwen3-235B            & 0.22 & 0.11    & 0.88 \\
Qwen3-Next            & 0.15 & --    & 1.50 \\
\bottomrule
\end{tabular}
\vspace{0.8em}
\caption{Unit API prices (USD per 1M tokens) used in our cost estimation.}
\label{tab:lite-version-cost}
\end{table}


\begin{table}[htbp]
\centering
\small
\begin{tabular}{lrrrrrrr}
\toprule
\textbf{Model} & \textbf{Search} & \textbf{MathHay} & \textbf{SWEBench} & \textbf{MCPBench} & \textbf{Tau2Bench} & \textbf{TerminalBench} & \textbf{Total} \\
\midrule
Gemini-2.5-Flash & \$193 & \$70.9 & \$12719 & \$122 & \$62.2 & \$826 & \$13993 \\
DeepSeek-R1 & \$5188 & \$157 & \$3675 & \$492 & \$202 & \$1505 & \$11218 \\
DeepSeek-V3.2 & \$369 & \$38.8 & \$7.80 & \$88.4 & \$54.4 & \$412 & \$970 \\
Qwen3-235B & \$1254 & \$41.4 & \$1286 & \$120 & \$55.5 & \$409 & \$3166 \\
Qwen3-Next & \$25.3 & \$9.52 & \$3.38 & \$21.2 & \$14.9 & \$154 & \$229 \\
\midrule
\textbf{Total} & \$7028 & \$317 & \$17692 & \$843 & \$389 & \$3307 & \$29576 \\
\bottomrule
\end{tabular}
\vspace{1em}
\caption{Cost for evaluating models under parallel scaling setting (USD)}
\label{tab:parallel-scaling-cost}
\end{table}

\begin{table}[htbp]
\centering
\small
\begin{tabular}{lrrrrrrr}
\toprule
\textbf{Model} & \textbf{Search} & \textbf{MathHay} & \textbf{SWEBench} & \textbf{MCPBench} & \textbf{Tau2Bench} & \textbf{TerminalBench} & \textbf{Total} \\
\midrule
Gemini-2.5-Flash & \$5588 & \$369 & \$2024 & \$568 & \$852 & \$1870 & \$11271 \\
DeepSeek-R1 & \$1267 & \$654 & \$892 & \$931 & \$1088 & \$782 & \$5614 \\
DeepSeek-V3.2 & \$902 & \$333 & \$191 & \$79.1 & \$52.3 & \$356 & \$1913 \\
Qwen3-235B & \$1370 & \$238 & \$436 & \$716 & \$162 & \$689 & \$3610 \\
Qwen3-Next & \$442 & \$219 & \$392 & \$152 & \$251 & \$529 & \$1985 \\
\midrule
\textbf{Total} & \$9568 & \$1814 & \$3935 & \$2445 & \$2405 & \$4225 & \$24392 \\
\bottomrule
\end{tabular}
\vspace{1em}
\caption{Cost for evaluating models under sequential scaling setting (USD)}
\label{tab:sequential-scaling-cost}
\end{table}

\begin{table}[htbp]
\centering
\small
\begin{tabular}{lrrrrrrr}
\toprule
\textbf{Model} & \textbf{Search} & \textbf{MathHay} & \textbf{SWEBench} & \textbf{MCPBench} & \textbf{Tau2Bench} & \textbf{TerminalBench} & \textbf{Total} \\
\midrule
Gemini-2.5-Pro & \$47.5 & \$18.6 & \$3253 & \$31.3 & \$15.2 & \$203 & \$3569 \\
GPT-5 & \$87.5 & \$14.2 & \$146 & \$21.0 & \$12.4 & \$60.9 & \$342 \\
Claude-Haiku-4.5 & \$304 & \$10.9 & \$312 & \$29.3 & \$14.1 & \$106 & \$776 \\
Claude-Sonnet-4.5 & \$1248 & \$40.9 & \$926 & \$129 & \$52.0 & \$376 & \$2772 \\
OpenAI-oss-120B & \$1.44 & \$1.49 & \$0.52 & \$0.50 & \$1.35 & \$0.40 & \$5.70 \\
DeepSeek-V3.2 & \$96.8 & \$10.0 & \$1.92 & \$22.0 & \$14.0 & \$102 & \$247 \\
Qwen3-Next & \$6.05 & \$2.42 & \$0.88 & \$5.51 & \$3.60 & \$38.2 & \$56.7 \\
\midrule
\textbf{Total} & \$1791 & \$98.5 & \$4640 & \$239 & \$113 & \$886 & \$7768 \\
\bottomrule
\end{tabular}
\vspace{1em}
\caption{Cost for evaluating models under general (default context) setting (USD)}
\label{tab:general-setting-cost}
\end{table}

% Table \ref{tab:full-version-cost} reports the API costs for each model if the full datasets from all prior benchmarks were used instead of the sampled subsets.

% \begin{table}[h]
% \centering
% \small
% \begin{tabular}{lccccccccc}
% \toprule
%  & \multicolumn{3}{c}{\textbf{Search}} & \multicolumn{2}{c}{\textbf{Code}} & \multicolumn{1}{c}{\textbf{Reason}} & \multicolumn{2}{c}{\textbf{Tool Call}} & \\
% \cmidrule(lr){2-4}\cmidrule(lr){5-6}\cmidrule(lr){7-7}\cmidrule(lr){8-9}
% \textbf{Model} & \textbf{BC} & \textbf{WV} & \textbf{M2W} & \textbf{SWEB} & \textbf{TerB} & \textbf{MH} & \textbf{TauB} & \textbf{MCPB} & \textbf{Total} \\
% \midrule
% Gemini-2.5-Flash &  &  &  & \$237 & \$20 &  & \$18.86 & \$8.28 &  \\
% Gemini-2.5-Pro &  &  &  & \$550 & \$80 &  & \$159.74 & \$28.81 &  \\
% GPT-5 &  &  &  & \$400 & \$80 &  & \$10.16 & \$48.68 &  \\
% Claude-Haiku 4.5 &  &  &  &  &  &  & \$3.45 & \$24.36 &  \\
% Claude-Sonnet 4.5 &  &  &  & \$550 & \$100 &  & \$16.39 & \$58.84 &  \\
% \midrule
% gpt-oss-120B &  &  &  &  &  &  & \$5.11 & \$3.23 &  \\
% Deepseek-R1 &  &  &  & \$400 & \$80 &  & \$8.87 & \$24.10 &  \\
% Deepseek V3.2 &  &  &  & \$300 & \$80 &  & \$2.19 & \$6.33 &  \\
% Qwen3-235B &  &  &  & \$100 & \$10 &  & \$8.86 & \$1.83 &  \\
% Qwen3-Next &  &  &  &  &  &  & \$22.98 & \$7.26 &  \\
% \midrule
% \textbf{Total} &  &  &  & \$5380 & \$800 &  & \$256.61 & \$211.72 &  \\
% \bottomrule
% \end{tabular}
% \vspace{1em}
% \caption{Cost for evaluating models on the \textbf{full} datasets from prior benchmarks instead of sampling subsets.}
% \label{tab:full-version-cost}
% \end{table}


% \subsection{Test-time scaling cost}

% We provide the cost of running these models for different target context lengths, broken down by domain.

% \begin{table}[h]
%     \centering
%     \small
%     \caption{API cost of different models on the \textbf{Search} domain under varying context lengths.}
%     \label{tab:search_cost_scale}
%     \begin{tabular}{lcccc}
%         \toprule
%         \textbf{Model} 
%             & \textbf{8k} 
%             & \textbf{16k} 
%             & \textbf{32k} 
%             & \textbf{64k} \\
%         \midrule
%         Gemini-2.5-Flash & \$x.xx & \$x.xx & \$x.xx & \$x.xx \\
%         gpt-oss-120B & \$x.xx & \$x.xx & \$x.xx & \$x.xx \\
%         Deepseek-v3.2 & \$x.xx & \$x.xx & \$x.xx & \$x.xx \\
%         Qwen3-235B & \$x.xx & \$x.xx & \$x.xx & \$x.xx \\
%         Qwen3-Next & \$x.xx & \$x.xx & \$x.xx & \$x.xx \\
%         \bottomrule
%     \end{tabular}
%     \vspace{0.5ex}
% \end{table}

% \begin{table}[h]
%     \centering
%     \small
%     \caption{API cost of different models on the \textbf{Code} domain under varying context lengths.}
%     \label{tab:code_cost_scale}
%     \begin{tabular}{lcccc}
%         \toprule
%         \textbf{Model} 
%             & \textbf{8k} 
%             & \textbf{16k} 
%             & \textbf{32k} 
%             & \textbf{64k} \\
%         \midrule
%         Gemini-2.5-Flash & \$x.xx & \$x.xx & \$x.xx & \$x.xx \\
%         gpt-oss-120B & \$x.xx & \$x.xx & \$x.xx & \$x.xx \\
%         Deepseek-v3.2 & \$x.xx & \$x.xx & \$x.xx & \$x.xx \\
%         Qwen3-235B & \$x.xx & \$x.xx & \$x.xx & \$x.xx \\
%         Qwen3-Next & \$x.xx & \$x.xx & \$x.xx & \$x.xx \\
%         \bottomrule
%     \end{tabular}
%     \vspace{0.5ex}
% \end{table}


% \begin{table}[h]
%     \centering
%     \small
%     \caption{API cost of different models on the \textbf{Reason} domain under varying context lengths.}
%     \label{tab:search_reason_scale}
%     \begin{tabular}{lcccc}
%         \toprule
%         \textbf{Model} 
%             & \textbf{1turn} 
%             & \textbf{2turns} 
%             & \textbf{4turns} 
%             & \textbf{8turns} \\
%         \midrule
%         Gemini-2.5-Flash & \$x.xx & \$x.xx & \$x.xx & \$x.xx \\
%         gpt-oss-120B & \$x.xx & \$x.xx & \$x.xx & \$x.xx \\
%         Deepseek-v3.2 & \$x.xx & \$x.xx & \$x.xx & \$x.xx \\
%         Qwen3-235B & \$x.xx & \$x.xx & \$x.xx & \$x.xx \\
%         Qwen3-Next & \$x.xx & \$x.xx & \$x.xx & \$x.xx \\
%         \bottomrule
%     \end{tabular}
%     \vspace{0.5ex}
% \end{table}

% \begin{table*}[ht]
% \centering
% \scriptsize
% \begin{tabular}{l*{10}{c}}
% \toprule
% \multirow{2}{*}{\textbf{Models}} 
%   & \multicolumn{4}{c}{\textbf{Parallel Scaling}} 
%   & \multicolumn{6}{c}{\textbf{Sequential Scaling}} \\
% \cmidrule(lr){2-5}\cmidrule(lr){6-11}
%  & \textbf{Best@1} & \textbf{Best@2} & \textbf{Best@4} & \textbf{Best@8}
%  & \textbf{4K} & \textbf{8K} & \textbf{12K} & \textbf{16K} & \textbf{22K} & \textbf{32K} \\
% \midrule
% gpt-oss-120B     & \$5.11  & \$10.01 & \$20.65 & \$41.72
%                  & \$1.41  & \$5.41  & \$8.54  & \$49.91 & \$37.93 & \$172.15 \\
% Qwen3-Next       & \$22.98 & \$34.36 & \$73.56 & \$115.13
%                  & \$1.99  & \$4.26  & \$6.27  & \$14.17 & \$24.34 & \$48.05 \\
% Qwen3-235B       & \$8.86  & \$18.15 & \$36.57 & \$81.08
%                  & \$1.62  & \$10.47 & \$7.07  & \$31.77 & \$42.13 & \$88.63 \\
% DeepSeek-V3.2    & \$2.19  & \$3.43  & \$6.10  & \$12.09
%                  & \$0.33  & \$1.36  & \$4.47  & \$9.09  & \$19.76 & \$59.56 \\
% Gemini-2.5-Flash & \$18.86 & \$34.04 & \$63.41 & \$107.14
%                  & \$2.56  & \$12.84 & \$31.37 & \$157.91 & \$129.85 & \$382.52 \\
% \midrule
% \textbf{Total}   & \textbf{\$58.00} & \textbf{\$99.99} & \textbf{\$200.29} & \textbf{\$357.16}
%                  & \textbf{\$7.91}  & \textbf{\$34.34} & \textbf{\$57.72}  & \textbf{\$262.85} 
%                  & \textbf{\$254.01} & \textbf{\$750.91} \\
% \bottomrule
% \end{tabular}
% \caption{$\tau^2$-Bench Parallel and Sequential Scaling Cost.}
% \label{tab:tau2bench_parallel_sequential_cost}
% \end{table*}



% \begin{table*}[t]
% \centering
% \small
% \resizebox{\linewidth}{!}{%
% \begin{tabular}{lcccc*{15}{c}}
% \toprule
% \multirow{2}{*}{\textbf{Model}}
%   & \multicolumn{4  }{c}{\textbf{Parallel Scaling}} 
%   & \multicolumn{5}{c}{\textbf{Sequential Scaling (1-Server)}} 
%   & \multicolumn{5}{c}{\textbf{Sequential Scaling (2-Server)}} 
%   & \multicolumn{5}{c}{\textbf{Sequential Scaling (3-Server)}} \\
% \cmidrule(lr){2-5}\cmidrule(lr){6-10}\cmidrule(lr){11-15}\cmidrule(lr){16-20}
%  & \textbf{Best@1} & \textbf{Best@2} & \textbf{Best@4} & \textbf{Best@8}
%  & \textbf{8k} & \textbf{16k} & \textbf{24k} & \textbf{32k} & \textbf{48k}
%  & \textbf{16k} & \textbf{24k} & \textbf{32k} & \textbf{48k} & \textbf{64k}
%  & \textbf{24k} & \textbf{32k} & \textbf{48k} & \textbf{64k} & \textbf{96k} \\
% \midrule
% OpenAI-oss-120B
%  & \$3.23 & \$6.45 & \$12.97 & \$26.17
%  & \$0.20 & \$0.23 & \$0.55 & \$2.35 & \$10.41
%  & \$0.12 & \$0.59 & \$0.36 & \$1.10 & \$2.71
%  & \$0.10 & \$0.14 & \$2.50 & \$2.77 & \$0.61 \\
% Qwen3-Next
%  & \$7.26 & \$14.31 & \$23.88 & \$51.11
%  & \$0.30 & \$0.44 & \$1.80 & \$3.51 & \$8.07
%  & \$0.22 & \$0.78 & \$3.10 & \$4.92 & \$7.22
%  & \$0.17 & \$0.48 & \$2.12 & \$3.88 & \$1.22 \\
% Qwen3-235B
%  & \$1.83 & \$3.56 & \$7.03 & \$14.00
%  & \$0.19 & \$1.31 & \$10.39 & \$24.65 & \$28.75
%  & \$0.12 & \$0.45 & \$12.19 & \$43.95 & \$26.82
%  & \$0.10 & \$0.68 & \$22.52 & \$73.28 & \$0.39 \\
% DeepSeek-V3.2
%  & \$6.33 & \$12.40 & \$20.57 & \$43.33
%  & \$0.22 & \$0.30 & \$2.57 & \$8.51 & \$16.95
%  & \$0.15 & \$1.05 & \$5.19 & \$4.24 & \$13.30
%  & \$0.12 & \$0.41 & \$3.31 & \$13.45 & \$1.23 \\
% Gemini-2.5-Flash
%  & \$8.28 & \$16.21 & \$33.47 & \$64.07
%  & \$0.38 & \$0.49 & \$6.29 & \$24.90 & \$46.74
%  & \$0.27 & \$0.68 & \$7.73 & \$21.40 & \$33.37
%  & \$0.26 & \$0.70 & \$3.24 & \$35.55 & \$1.72 \\
% \midrule
% \textbf{Total}
%  & \$\textbf{26.93} & \$\textbf{52.93} & \$\textbf{97.92} & \$\textbf{198.68}
%  & \$\textbf{1.29} & \$\textbf{2.77} & \$\textbf{21.60} & \$\textbf{63.92} & \$\textbf{110.92}
%  & \$\textbf{0.88} & \$\textbf{3.55} & \$\textbf{28.57} & \$\textbf{75.61} & \$\textbf{83.42}
%  & \$\textbf{0.75} & \$\textbf{2.41} & \$\textbf{33.69} & \$\textbf{128.93} & \$\textbf{5.17} \\
% \bottomrule
% \end{tabular}%
% }
% \vspace{0.2cm}
% \caption{MCP-Bench Parallel and Sequential Scaling Experiment Costs.}
% \label{tab:mcp-bench-parallel-seq-costs}
% \end{table*}



\newpage
% \begin{tcolorbox}[
%   colback=gray!5,
%   colframe=gray!70,
%   title=\textbf{Case Study: Retrieving the Latest HuggingFace Model for Text Classification},
%   fonttitle=\bfseries\small,
%   fontupper=\ttfamily\scriptsize,
%   breakable
% ]

% \textbf{Task.}
% Identify the most recent pre-trained language model on HuggingFace suitable for text classification.

% \vspace{4pt}
% \hrule
% \vspace{6pt}

% \textbf{Plain Web Search Baseline (6 turns).}
% \begin{verbatim}
% Turn 1: <search> latest Hugging Face pre-trained language 
%         model text classification 2024 </search>
% Turn 2: <search> Hugging Face text classification models 
%         sorted by recent trending 2024 </search>
% Turn 3: <search> Hugging Face newest text classification 
%         models January 2025 recently released </search>
% ...
% Turn 6: <answer> tabularisai/multilingual-sentiment-analysis
%         (fine-tuned DistilBERT-base-multilingual-cased)
%         </answer>
% \end{verbatim}

% \vspace{4pt}
% \hrule
% \vspace{6pt}

% \textbf{General agent response with Specialized Tools (4 turns).}
% \begin{verbatim}
% Turn 1: Hugging_Face__search-models
%         {"tags": "text-classification", "limit": 15}
%         -> [{"id": "FutureMa/Qwen3-4B-Evasion", "downloads": 35},
%             {"id": "BAAI/bge-reranker-v2-m3", "downloads": 2807076}, ...]

% Turn 2: search__web_search "ModernBERT Hugging Face 2025"
%         -> "ModernBERT... 8,192-token context, 139M/395M params,
%            trained on 2T tokens"

% Turn 3: Hugging_Face__get-model-info
%         {"model_id": "answerdotai/ModernBERT-base"}
%         -> Full model card and architecture details
% \end{verbatim}

% \end{tcolorbox}

\section{General AgentBench Implementation Details}

\subsection{Tool Management}

\paragraph{Tool Registration}
The General AgentBench follows the Model Context Protocol (MCP) architecture with Host-Client-Server design. During server connection, the \textit{BenchmarkHost} creates transport connections (STDIO/HTTP), initializes \textit{ClientSession}, and discovers tools via \textit{list\_tools()}. All tools are registered to a global \textit{tool\_to\_client} routing map. 

%Distraction masking is applied at schema generation time, not during registration: when \textit{get\_filtered\_tools\_schema()} is called, it first identifies required tools for the task, then randomly samples \textit{distraction\_count} tools from the remaining pool. Setting \textit{distraction\_count=-1} includes all 301 available tools. During tool invocation, \textit{route\_tool\_call()} looks up the routing map, extracts the original tool name from MCP-Bench format, and routes to the appropriate client.

\paragraph{Tool Schema}
Each tool follows OpenAI function-calling format with \textit{name}, \textit{description}, and \textit{parameters} (JSON Schema), as shown in Box~\ref{box:tool-schema}. Tools use Bedrock-compatible naming requirements ([a-zA-Z0-9\_-]+): MCP-Bench format uses \textit{ServerName\_\_tool\_name} (e.g., \textit{BioMCP\_\_think}), while Tau2 uses \textit{domain\_tool\_name} (e.g., \textit{airline\_book\_reservation}). The complete toolset in \textit{Host} contains \textbf{301 tools} across 35 servers, including BioMCP (34), Scientific\_Computing (26), Medical\_Calculator (22), NASA\_Data (21), NixOS (18), Unit\_Converter (16), tau2-retail (15), tau2-airline (14), tau2-telecom (13), and others. 

%The full JSON schema consumes approximately \textbf{78,083 tokens} (GPT-4 tokenization).

\begin{tcolorbox}[
  colback=gray!5,
  colframe=gray!70,
  title=\textbf{An example of tool schema: get-collection-info function from Huggingface server},
  fonttitle=\bfseries\small,
  label=box:tool-schema,
  breakable
]
\label{{box:tool-schema}}
{\scriptsize
\begin{verbatim}

{
    "type": "function",
    "function": {
      "name": "Hugging_Face__get-collection-info",
      "description": "Get detailed information about a specific collection",
      "parameters": {
        "type": "object",
        "properties": {
          "namespace": {
            "type": "string",
            "description": "The namespace of the collection (user or organization)"
          },
          "collection_id": {
            "type": "string",
            "description": "The ID part of the collection"
          }
        },
        "required": [
          "namespace",
          "collection_id"
        ]
      }
    }
}
\end{verbatim}
}



\end{tcolorbox}

\paragraph{Description Compression and Minimal Strategy}
To reduce context consumption, we implement multi-level compression. The \textit{--compress-tools} option truncates descriptions to target number of characters and removes parameter defaults, achieving 18.6\% token reduction. If the description of the first sentence is truncated in the middle, we will keep the first sentence for tool loading.. For self-choice evaluation, we adopt the \textit{--minimal-tools} strategy: converting JSON schema to plain text format (\textit{tool\_name(params): short\_desc}), retaining only tool names, 50-character descriptions, and parameter names. This achieves \textbf{90.1\% token reduction}, saving approximately 70K tokens. 

%Additionally, minimal mode truncates trajectory outputs to 500 characters and limits task descriptions to 8,000 characters, enabling efficient self-choice evaluation within context budgets.



\subsection{Benchmark Integration Details}

\paragraph{User Simulator Implementation in Tau2Bench}
Tau2Bench requires multi-turn conversational interactions between the agent and a simulated user. We implement this through \textit{Tau2UserSimulatorAdapter}, which wraps the original Tau2 \textit{UserSimulator}. During each conversation turn, the adapter converts the agent's response to Tau2's \textit{AssistantMessage} format, calls \textit{generate\_next\_message(message, state)} to obtain the user response, and maintains conversation state across turns. User-side tool execution (e.g., checking order status) is routed through MCP internal callbacks, which ensures proper environment initialization. 
%The user simulator LLM is configured independently via \textit{--user-model} and \textit{--user-temperature} (default 0.0 for deterministic behavior), with \textit{--simulation-seed} ensuring reproducibility aligned with the original tau2-bench framework.

\paragraph{Docker Environment Interaction}
Both Terminal Bench and SWEBench employ Docker Container Bridge Mode for isolated task execution. The MCP servers run as persistent host processes, interacting with task containers via \textit{docker exec}. Public tools exposed to agents include \textit{execute\_bash}, \textit{read\_file}, \textit{write\_file}, and \textit{finish}(SWE-Bench). 
% Internal tools prefixed with \textit{\_\_} (invisible to agents) enable container lifecycle management: \textit{\_\_switch\_container} cleans up the previous container and starts a new one using Docker Compose, while \textit{\_\_run\_tests} executes the pytest harness. 
Each container receives a unique UUID-suffixed project name to support parallel experiments. This MCP Process Mode design avoids server restart overhead across tasks while maintaining complete environment isolation through Docker.

\paragraph{Evaluators}
The General Agent system adopts a native evaluator delegation strategy, directly invoking each benchmark's original evaluation code rather than reimplementing evaluation logic. For Tau2Bench, we compute rewards as the product of environment state matching (DB\_CHECK), action sequence validation, and communication checks. SWEBench and Terminal Bench use Docker Bridge Mode: MCP servers run as persistent host processes, executing \textit{pytest} harnesses in task containers via \textit{docker exec}, then parsing results using toolsets in original benchmark implementation. Search benchmarks (BrowseComp, Mind2Web, WebVoyager, GAIA) use their original \texttt{eval\_scripts} with LLM-based semantic equivalence or rubric evaluation. All evaluators produce binary rewards (0/1) except MCPBench (continuous 0-1). 

% This delegation approach ensures complete result comparability with published benchmark scores.






% \section{Details of the implementation.}

\subsection{Search Agent}

The search agent is currently a prompt only agent, meaning it doesn't utilize the API for tool calls. Tools are put into the prompt and parsed. It supports the following tools:
\begin{itemize}
\item \textbf{search}: search the web for information if you consider you lack some knowledge.
\item \textbf{answer}: output the final answer if you consider you are able to answer the question. The answer should be short and concise. No justification is needed.
\item \textbf{summary}: summarize the history turns. Reflect the search queries and search results in you history turns, and keep the information you consider important for answering the question and generating your report. Still keep the tag structure, $...$ The history turn information for your subsequent turns will be updated according to this summary action.
\item \textbf{think}: thinking process
\end{itemize}
There are two different prompts, one for internal thinking with no think tool, and one with external thinking with a think tool. zThe result of the search appends the search output within information tags which is fed back into the agent. 

\subsection{Coding Agent - OpenHands}

The coding agent OpenHands utilizes LiteLLM API as a unified API to call all language models. Tools are called through the API rather than in the prompt.
\begin{itemize}
\item \textbf{execute bash}: Execute a bash command in the terminal within a persistent shell session.
\item \textbf{str replace editor}: Custom editing tool for viewing, creating and editing files in plain-text format
\item \textbf{finish}: Signals the completion of the current task or conversation.
\item \textbf{task tracker}: This tool provides structured task management capabilities for development workflows.
It enables systematic tracking of work items, progress monitoring, and efficient
organization of complex development activities.
\item \textbf{think}: Use the tool to think about something. It will not obtain new information or make any changes to the repository, but just log the thought. Use it when complex reasoning or brainstorming is needed.
\end{itemize}

% \subsection{Math Agent - MathHay}


\subsection{Tool-calling Agent}
\subsubsection{$\tau^2$-Bench}
\paragraph{Framework} $\tau^2$-bench implements a simulation framework for evaluating customer service agents across various domains. This framework invloves \textbf{1) customer service agent}, the agent providing help with customer demands and \textbf{2) user agent}, giving responses to simulate human-like feedbacks in multi-round conversation. 

\paragraph{Evaluation} $\tau^2$-Bench uses binary evaluation (0/1) by comparing the agent's final environment state against a gold standard through environment replay. The reward is the product of multiple dimensions, covering database state, environment assertions, action sequences, and communication. A task succeeds only when all dimensions achieve 1.0, requiring exact database state matching via hash comparison and complete satisfaction of all assertions and action requirements. 

\paragraph{Datasets}
% types
% lite sampling
$\tau^2$-Bench provides 3 domains: airline, retail and telecom to simulate customer service conversation in real-life scenarios. Each domain contains a policy the agent must follow, a set of tools that agent can use and a set of tasks to evaluate the agent's performance. 

The original test set contains 50 tasks for airline domain, 114 tasks for retail and 114 tasks for telecom(midium version). Considering the cost of agents, We selected 10 tasks for airline domain, 20 tasks for retail domain and 20 tasks for telecom domain in our sampled test set accoridng to conversation length distribution. We first run the benchmark on gpt-oss-120B, Claude Sonnet 4.5 and Gemini-2.5-Pro models and get an estimation of conversation length for each task(measured by the length of last round input of customer service agent). Then we sample the data uniformly given the estimated length distribution to ensure the sampled dataset is representative in converastion length.


\paragraph{General Settings}
We choose gpt-oss-120B model as user agent simulator model. The temperature of user agent is set to 0.0 for stable human response simulation. The temperature of customer agent is set to 0.70 to encourage model to try different ways in solving problems. Please refer to Table~\ref{tab:tau2bench-hyperparams} for detailed hyperparameter settings. 
%$\tau^2$-Bench manage API calls using LiteLLM and we further add support for OpenAI platform and 

\begin{table}[]
\centering
% \scriptsize
\begin{tabular}{ll}
\toprule
\textbf{Parameter} & \textbf{Value} \\
\midrule
DEFAULT\_SEED & 300 \\
DEFAULT\_MAX\_STEPS & 2000 \\
DEFAULT\_MAX\_ERRORS & 10 \\
DEFAULT\_MAX\_CONCURRENCY & 3 \\
DEFAULT\_NUM\_TRIALS & 1 \\
DEFAULT\_AGENT\_IMPLEMENTATION & llm\_agent \\
DEFAULT\_USER\_IMPLEMENTATION & user\_simulator \\
DEFAULT\_LLM\_USER & gpt-oss-120B-1:0 \\
DEFAULT\_LLM\_TEMPERATURE\_AGENT & 0.70 \\
DEFAULT\_LLM\_TEMPERATURE\_USER & 0.00 \\
\bottomrule
\end{tabular}
\vspace{0.3cm}
\caption{$\tau^2$-Bench benchmark hyperparameters.}
\label{tab:tau2bench-hyperparams}
\end{table}


\paragraph{Parallel Scaling Settings}
% metric calculation bestk
% 为什么选8次
% metrics 看的是什么指标
For parallel scaling, we run the benchmark 8 times for each model and record past@$K$ results. We count results with rewards=1.0 as successful results. For each task, if the task is successfully solved at least once in first $k$ times running, past@$K$ of this task will be counted as 1. The final past@$K$ results are the average on all 50 tasks.


\paragraph{Sequential Scaling Settings}
% stop/extension strategy
% prompt
% budget selection
% metric calculation
In sequential scaling, we control the length of conversation by dynamically adding stop/extension prompts. 
% Please refer to Box~\ref{box:prompt-stop} and Box~\ref{box:prompt-extend} for detailed prompts. 
Since in $\tau^2$-Bench framework, the user agent controls the ending of the conversation by sending end-signals, we implement the dynamic strategy in user agent simulator. Our strategy dynamically injects stop or extension prompts based on both agent intent and token usage. It consists of two parts: \textbf{1) Add STOP prompt.} The simulator monitors cumulative context tokens against a configurable budget and intercepts user messages to detect explicit end-signals(signal of transferring to human operator, conversation out of scope or simply stop). If the budget is reached, it emits a soft stop by appending a directive STOP\_PROMPT and defers the actual end-signal to the next turn via a ready-stop flag. Notice that the stopping strategy will allow the customer service agent act for another round and force user simulator to terminate the conversation by sending end-signal in the next turn. \textbf{2) Add EXTEND prompt.} If the user attempts to stop before the budget is exhausted, we will cleans the end-signal and  appends an EXTEND\_PROMPT that nudges the agent to deepen reasoning, verify outputs, and seek missing evidence. 

\begin{tcolorbox}[
  colback=gray!5,
  colframe=gray!70,
  title=\textbf{Stop Prompt},
  fonttitle=\bfseries\small,
  breakable
]
**CRITICAL: You MUST provide your final answer immediately. Do NOT perform any more tool calling or reasoning. Return the final answer under the required format NOW.**
\end{tcolorbox}

\begin{tcolorbox}[
  colback=gray!5,
  colframe=gray!70,
  title=\textbf{Extension Prompt},
  fonttitle=\bfseries\small,
  breakable
]
Before finalizing your answer, take additional time to verify your reasoning, consider alternative approaches, and search for any missing information that could strengthen your response.
\end{tcolorbox}



\subsubsection{MCP-Bench}
\paragraph{Framework.}
For each benchmark instance, MCP-Bench presents the agent with a fuzzy natural language task description and a large MCP tool ecosystem consisting of relevant servers plus distractor servers. The agent then interacts with this environment in a multi-round plan–execute loop: at each step it selects one or more tools, constructs structured arguments that respect the MCP schema, and invokes these tools to obtain intermediate results. The raw tool outputs are optionally compressed into shorter summaries to control context length while preserving information needed for downstream decisions. These summaries, together with the original outputs and previous actions, are appended to an execution trace that forms the agent’s internal state. The agent iteratively updates this state, revises its plan, and decides whether to continue calling tools or terminate. Once it judges the task to be complete, it produces a final natural language answer conditioned on the accumulated execution trace.

\paragraph{Evaluation.}
The evaluation pipeline combines rule-based execution checks with rubric-based LLM-as-a-Judge scoring, and summarizes performance along four axis-level scores. Rule-based metrics are computed directly from the tool-call trace, covering valid tool naming, input-schema adherence, runtime success, and dependency compliance. The LLM judge is then prompted with the fuzzy task description, the hidden concrete task specification and dependency analysis, the summarized execution trajectory, and the final answer, and assigns rubric-based scores along higher-level dimensions. There are four axis scores:
\textbf{1) Schema Understanding Score} includes Valid Tool Name Rate ($s_{\text{name}})$, Schema Compliance ($s_{\text{schema}}$) and Execution Success ($ s_{\text{execution}}$). Captures the agent’s low-level robustness in handling tool interfaces, aggregating valid tool name rate, schema compliance rate, and execution success rate into a single score that reflects how reliably the agent issues well-formed, executable calls.
\textbf{2) Task Completion Score} includes Task Fulfillment ($s_{\text{fulfillment}}$) and Information Grounding ($s_{\text{grounding}}$). Measures whether the agent actually solves the task, based on LLM-judge sub-dimensions such as task fulfillment, coverage of required subtasks, information grounding in tool outputs, and overall answer relevance.
\textbf{3) Tool Usage Score} includes Tool Appropriateness ($s_{\text{appropriateness}}$) and Parameter Accuracy ($s_{\text{parameter}}$). Evaluates how appropriately the agent selects and parameterizes tools, combining rubric scores for tool appropriateness (matching tools to subtasks) and parameter accuracy (correct and sufficient arguments for each call).
\textbf{4) Planning Effectiveness Score} includes Dependency Awareness($s_{\text{dependency}}$) and Parallelism and Efficiency($s_{\text{efficiency}}$). Assesses the structural quality of multi-round execution, including dependency awareness: respecting inter-tool ordering and data flow; and parallelism and efficiency: avoiding redundant calls and exploiting safe concurrency when possible. Considering sequential scaling settings, we excluded parallelism and efficiency score and only take dependency awareness into account.

Each score is normalized to $[0,1]$ and the final benchmark score for a model on a given setting is given as:

\begin{equation}
    S_{\text{overall}}
    = \frac{1}{2} \left(
        \frac{
        s_{\text{name}}+
        s_{\text{schema}}+
        s_{\text{execution}}
        }{3} + 
        \frac{
        s_{\text{fulfillment}}+
        s_{\text{grounding}}+
        s_{\text{appropriateness}}+
        s_{\text{parameter}}+
        s_{\text{dependency}}+
        s_{\text{efficiency}}
        }{5}
      \right).
\end{equation}

\paragraph{Datasets}
MCP-Bench requires the agent to solve the problem using 1 or 2 or 3 server, thus dividing the dataset into 3 subsets accordingly. Considering the tradeoff between data variaty and cost, we select 1 problem per server combination, which totals 52 tasks with 28 1-server tasks, 15 2-server tasks and 9 3-server tasks. 

\begin{table}[t]
\centering
% \scriptsize
\begin{tabular}{lll}
\toprule
\textbf{Category} & \textbf{Key} & \textbf{Value} \\
\midrule
\multirow{11}{*}{execution} & task\_timeout & 5000 \\
& task\_retry\_max & 1 \\
& retry\_delay & 5 \\
& compression\_retries & 1 \\
& max\_execution\_rounds & 2000 \\
& server\_semaphore\_limit & 20 \\
& context\_budget & null \\
& content\_summary\_threshold & 1000 \\
& content\_truncate\_length & 4000 \\
& error\_truncate\_length & 1000 \\
& error\_display\_prefix & 200 \\
\midrule
\multirow{3}{*}{evaluation} & judge\_stability\_runs & 5 \\
& judge\_model.name & gpt-oss-120B \\
& judge\_model.provider & LiteLLM \\
\midrule
\multirow{9}{*}{llm} & temperature & 0.7 \\
& json\_retry\_groups & 5 \\
& token\_reduction\_factors & [0.9, 0.8, 0.7] \\
& min\_tokens & 1000 \\
& token\_increment & 1000 \\
& format\_conversion\_tokens & 8000 \\
& planning\_tokens & 12000 \\
& summarization\_max\_tokens & 10000 \\
& user\_prompt\_max\_length & 30000 \\
\bottomrule
\end{tabular}
\vspace{0.2cm}
\caption{MCP-Bench configuration summary.}\label{tab:mcpbench_config}
\end{table}

\paragraph{General Settings} 
Table~\ref{tab:mcpbench_config} shows general settings for MCP-Bench. In case sequential scaling is early stopped by max\_execution\_rounds, we set its upper bounds well beyond the conversation rounds in sequential scaling. And we choose gpt-oss-120B as evaluator (different from o4-mini, gpt-4o and gpt-4o-mini) and run 5 times for each task to get a stable evaluation. Please refer to Table~\ref{app:tab:mcpbench-config} for detailed settings.

\begin{table*}[]
\centering
% \scriptsize
\begin{tabular}{lll}
\toprule
Category & Key & Value \\
\midrule
\multirow{6}{*}{mcp.connection} & http\_timeout & 60 \\
& tool\_discovery\_timeout & 10 \\
& server\_startup\_timeout & 30 \\
& health\_check\_timeout & 2 \\
& process\_wait\_timeout & 5 \\
& batch\_timeout & 60 \\
\midrule
\multirow{11}{*}{execution} & task\_timeout & 5000 \\
& task\_retry\_max & 1 \\
& retry\_delay & 5 \\
& compression\_retries & 1 \\
& max\_execution\_rounds & 2000 \\
& server\_semaphore\_limit & 20 \\
& context\_budget & null \\
& content\_summary\_threshold & 1000 \\
& content\_truncate\_length & 4000 \\
& error\_truncate\_length & 1000 \\
& error\_display\_prefix & 200 \\
\midrule
\multirow{3}{*}{evaluation} & judge\_stability\_runs & 5 \\
& judge\_model.name & gpt-oss-120B \\
& judge\_model.provider & LiteLLM \\
\midrule
\multirow{9}{*}{llm} & temperature & 0.7 \\
& json\_retry\_groups & 5 \\
& token\_reduction\_factors & [0.9, 0.8, 0.7] \\
& min\_tokens & 1000 \\
& token\_increment & 1000 \\
& format\_conversion\_tokens & 8000 \\
& planning\_tokens & 12000 \\
& summarization\_max\_tokens & 10000 \\
& user\_prompt\_max\_length & 30000 \\
\midrule
\multirow{10}{*}{data\_collection} & individual\_timeout & 30 \\
& batch\_timeout & 60 \\
& max\_retries & 2 \\
& retry\_delay\_base & 3 \\
& retry\_delay\_multiplier & 2 \\
& batch\_retry\_delay\_base & 5 \\
& batch\_retry\_delay\_multiplier & 3 \\
& default\_http\_port & 3000 \\
& tool\_description\_truncate & 150 \\
& (no\_cache\_fields\_here) & - \\
\midrule
\multirow{2}{*}{task\_generation} & tasks\_per\_combination & 1 \\
& generation\_max\_retries & 1 \\
\midrule
\multirow{2}{*}{dependency\_extraction} & required\_support\_count & 3 \\
& min\_support\_count & 2 \\
\midrule
\multirow{2}{*}{server\_selection} & selection\_tokens & 8000 \\
& tool\_sample\_count & 3 \\
\bottomrule
\end{tabular}
\caption{MCP-Bench configuration summary.}\label{app:tab:mcpbench-config}
\end{table*}


\paragraph{Parallel Scaling Settings}
Similar to $\tau^2$-Bench, we run MCP-Bench for 4 times on each model and record past@$K$ results. For each task, we choose the highest $S_{\text{overall}}$ in first $k$ runs as the past@$K$ score for this task. Then we average past@$K$ scores from 52 tasks as final past@$k$ score. 


\paragraph{Sequential Scaling Settings} 
In sequential scaling, we control conversation length by dynamically adding stop/extension prompts during multi-round execution. Stop prompt and extension prompt are the same as $\tau^2$-Bench. 
% Please refer to Box~\ref{box:prompt-stop} and Box~\ref{box:prompt-extend} for detailed prompts. 

In MCP-Bench, the decision loop is implemented inside the executor. Our strategy injects STOP or EXTEND prompts based on agent intent and token usage tracked per round. It consists of two parts: \textbf{1) Add STOP prompt.} Before each planning call, the executor estimates this round’s prompt tokens against a configurable context\_budget. If the estimated prompt for the current round would exceed the budget, it appends a STOP\_PROMPT to the planning prompt, sets force\_stop, and prevents new tool executions (planned\_tools cleared). When the agent signals stop without a budget, the executor terminates the loop. \textbf{2) Add EXTEND prompt.} If the agent attempts to stop while the budget still permits another round, the executor estimates the next round’s cost as current\_round\_tokens plus the EXTEND\_PROMPT tokens. If within budget, it sets an extend\_next\_round flag, flips should\_continue to true for this turn, and injects EXTEND\_PROMPT into the next round’s planning prompt, nudging the agent to deepen reasoning, verify outputs, and seek missing evidence. 

For token budget choices, we first use gpt-oss-120B to estimate context length of the task (measured by the context length of the last round conversation input) and we find the context length varies across 1-server, 2-server and 3server tasks. Generally, for 1-server tasks, the context length is shorter than that of 3-server due to task complexity. If we select fixed length for all 3 servers, some choices of small token budgets will stop the conversation in complex tasks even before the first round conversation ends; some larger budgets will still extend the conversation even though the problem has been solved with shorter context. Therefore, we select different scale in sequential scaling: 8k-32k for 1-server, 16k-64k for 2-server and 24k-96k for 3-server.

% \subsection{MCP Agent}
% \subsubsection{$\tau^2$-Bench}
% $\tau^2$-bench is a dual-control simulation framework for evaluating customer service agents in 3 scenarios: airline, retail and telecom. There are two roles in this system: an user for human-like response and an agent for customer service. The simulation environment is from databases based on local json files, such as user info and flight info. The bench also designed several tools for each specific domains locally. The evaluation is both rule-based and LLM-as-Judge. $\tau$-Bench use LiteLLM to manage API calls. 


% \begin{table}[h!]
% \centering
% \small
% \begin{tabular}{llp{6cm}c}
% \toprule
% \textbf{Domain} & \textbf{Tool Type} & \textbf{APIs} & \textbf{Total} \\
% \midrule

% \multirow{3}{*}{\textbf{Airline}} 
% & Agent Tools & Book Reservation, Calculate, Cancel Reservation,...
% %Get Reservation Details, Get User Details, List All Airports, Search Direct Flight, Search Onestop Flight, Send Certificate, Transfer To Human Agents, Update Reservation Baggages, Update Reservation Flights, Update Reservation Passengers, Get Flight Status 
% & 14 \\
% & User Tools  & --- & 0 \\

% \midrule

% \multirow{3}{*}{\textbf{Retail}} 
% & Agent Tools & Calculate, Cancel Pending Order, Exchange Delivered Order Items, ...
% %Find User Id By Name Zip, Find User Id By Email, Get Order Details, Get Product Details, Get User Details, List All Product Types, Modify Pending Order Address, Modify Pending Order Items, Modify Pending Order Payment, Modify User Address, Return Delivered Order Items, Transfer To Human Agents 
% & 15 \\
% & User Tools  & --- & 0 \\

% \midrule

% \multirow{5}{*}{\textbf{Telecom}} 
% & Agent Tools & Get Customer By Phone, Get Customer By Id, Get Customer By Name, Get Details By Id, 
% %Suspend Line, Resume Line, Get Bills For Customer, Send Payment Request, Get Data Usage, Enable Roaming, Disable Roaming, Transfer To Human Agents, Refuel Data 
% & 13 \\
% & User Tools & Check Status Bar, Check Network Status, Check Network Mode Preference, 
% %Set Network Mode Preference, Run Speed Test, Toggle Airplane Mode, Check Sim Status, Reseat Sim Card, Toggle Data, Toggle Roaming, Check Data Restriction Status, Toggle Data Saver Mode, Check Apn Settings, Set Apn Settings, Reset Apn Settings, Check Wifi Status, Toggle Wifi, Check Wifi Calling Status, Toggle Wifi Calling, Check Vpn Status, Connect Vpn, Disconnect Vpn, Check Installed Apps, Check App Status, Check App Permissions, Grant App Permission, Can Send Mms, Reboot Device, Check Payment Request, Make Payment 
% & 30 \\

% \bottomrule
% \end{tabular}
% \vspace{0.5em}
% \caption{Overview of Available Tools across Domains in $\tau^2$-Bench}

% \end{table}


% \subsubsection{MCP-Bench}
% MCP generally follows: 1. LLMs send request 2. MCP server receives request and processes request 3. Return execution results to LLMs with standard format.For MCP-Bench, there are 28 MCP servers and 246 tools in total. Table \ref{tab:mcp-bench-tools} lists server name and their tools. Evaluations are based on LLM-as-Judge. LiteLLM was added to manage API calls.

% %But the \hyperlink{https://clinicaltrialsapi.cancer.gov/signin}{login system on National Cancer Institute(NIH)} website  seems to be broken(?).

% \begin{table}[h]
% \centering
% \small
% \begin{tabular}{lp{7cm}r}
% \toprule
% \textbf{Server} & \textbf{Tools (partial)} & \textbf{Total} \\
% \midrule

% BioMCP & search, fetch, think, article\_searcher, % trial\_searcher, trial\_getter, trial\_protocol\_getter, trial\_references\_getter, trial\_outcomes\_getter, trial\_locations\_getter, variant\_searcher, variant\_getter, alphagenome\_predictor, gene\_getter, disease\_getter, drug\_getter, nci\_organization\_searcher, nci\_organization\_getter, nci\_intervention\_searcher, nci\_intervention\_getter, nci\_biomarker\_searcher, nci\_disease\_searcher, openfda\_adverse\_searcher, openfda\_adverse\_getter, openfda\_label\_searcher, openfda\_label\_getter, openfda\_device\_searcher, openfda\_device\_getter, openfda\_approval\_searcher, openfda\_approval\_getter, openfda\_recall\_searcher, openfda\_recall\_getter, openfda\_shortage\_searcher, openfda\_shortage\_getter
% & 35 \\

% Scientific Computing & create\_tensor, view\_tensor, delete\_tensor, add\_matrices, % subtract\_matrices, multiply\_matrices, scale\_matrix, matrix\_inverse, transpose, determinant, rank, compute\_eigen, qr\_decompose, svd\_decompose, find\_orthonormal\_basis, change\_basis, vector\_project, vector\_dot\_product, vector\_cross\_product, gradient, curl, divergence, laplacian, directional\_deriv, plot\_vector\_field, plot\_function
% & 26 \\

% Medical Calculator & egfr\_epi, egfr\_epi\_cr\_cys, bp\_children, bmi\_bsa\_calculator, % crcl\_cockcroft\_gault, map\_calculator, chads2\_vasc\_score, prevent\_cvd\_risk, corrected\_calcium, qtc\_calculator, wells\_pe\_criteria, ibw\_abw\_calculator, pregnancy\_calculator, revised\_cardiac\_risk\_index, child\_pugh\_score, steroid\_conversion, calculate\_mme, maintenance\_fluids, corrected\_sodium, meld\_3, framingham\_risk\_score, homa\_ir
% & 22 \\

% NASA Data & get\_astronomy\_picture\_of\_day, get\_asteroids\_feed, get\_asteroid\_lookup, browse\_asteroids, % get\_coronal\_mass\_ejection, get\_geomagnetic\_storm, get\_solar\_flare, get\_solar\_energetic\_particle, get\_magnetopause\_crossing, get\_radiation\_belt\_enhancement, get\_hight\_speed\_stream, get\_wsa\_enlil\_simulation, get\_notifications, get\_earth\_imagery, get\_earth\_assets, get\_epic\_imagery, get\_epic\_imagery\_by\_date, get\_epic\_dates, get\_exoplanet\_data, get\_mars\_rover\_photos, get\_mars\_rover\_manifest
% & 21 \\

% Paper Search & search\_arxiv, search\_pubmed, search\_biorxiv, search\_medrxiv, % search\_google\_scholar, search\_iacr, download\_arxiv, download\_pubmed, download\_biorxiv, download\_medrxiv, download\_iacr, read\_arxiv\_paper, read\_pubmed\_paper, read\_biorxiv\_paper, read\_medrxiv\_paper, read\_iacr\_paper, search\_semantic, download\_semantic, read\_semantic\_paper
% & 19 \\

% NixOS & nixos\_search, nixos\_info, nixos\_channels, nixos\_stats, % home\_manager\_search, home\_manager\_info, home\_manager\_stats, home\_manager\_list\_options, home\_manager\_options\_by\_prefix, darwin\_search, darwin\_info, darwin\_stats, darwin\_list\_options, darwin\_options\_by\_prefix, nixos\_flakes\_stats, nixos\_flakes\_search, nixhub\_package\_versions, nixhub\_find\_version
% & 18 \\

% Unit Converter & convert\_temperature, convert\_angle, convert\_length, convert\_energy, % convert\_force, convert\_pressure, convert\_power, convert\_speed, convert\_area, convert\_mass, convert\_volume, convert\_computer\_data, convert\_density, convert\_time, convert\_batch, list\_supported\_units
% & 16 \\

% Math MCP & add, subtract, multiply, division, % sum, mean, median, mode, min, max, floor, ceiling, round
% & 13 \\

% DEX Paprika & getNetworks, getNetworkDexes, getNetworkPools, getDexPools, % getPoolDetails, getTokenDetails, getTokenPools, getPoolOHLCV, getPoolTransactions, search, getStats
% & 11 \\

% Hugging Face & search-models, get-model-info, search-datasets, get-dataset-info, % search-spaces, get-space-info, get-paper-info, get-daily-papers, search-collections, get-collection-info
% & 10 \\

% Google Maps & search\_nearby, get\_place\_details, maps\_geocode, maps\_reverse\_geocode, % maps\_distance\_matrix, maps\_directions, maps\_elevation
% & 7 \\

% Game Trends & get\_steam\_trending\_games, get\_steam\_top\_sellers, get\_steam\_most\_played, get\_epic\_free\_games, % get\_epic\_trending\_games, get\_all\_trending\_games, get\_api\_health
% & 7 \\

% OSINT Intelligence & whois\_lookup, nmap\_scan, dnsrecon\_lookup, dnstwist\_lookup, % dig\_lookup, host\_lookup, osint\_overview
% & 7 \\

% National Parks & findParks, getParkDetails, getAlerts, getVisitorCenters, % getCampgrounds, getEvents
% & 6 \\

% Bibliomantic & i\_ching\_divination, bibliomantic\_consultation, get\_hexagram\_details, server\_statistics
% & 4 \\

% Weather Data & get\_current\_weather\_tool, get\_weather\_forecast\_tool, search\_locations\_tool, get\_live\_temp
% & 4 \\

% Car Price Evaluator & get\_car\_brands, search\_car\_price, get\_vehicles\_by\_type
% & 3 \\

% Huge Icons & list\_icons, search\_icons, get\_platform\_usage
% & 3 \\

% Metropolitan Museum & list-departments, search-museum-objects, get-museum-object
% & 3 \\

% OpenAPI Explorer & getApiOverview, getApiOperation
% & 2 \\

% Context7 & resolve-library-id, get-library-docs
% & 2 \\

% OKX Exchange & get\_price, get\_candlesticks
% & 2 \\

% Time MCP & get\_current\_time, convert\_time
% & 2 \\

% Call for Papers & get\_events
% & 1 \\

% FruityVice & get\_fruit\_nutrition
% & 1 \\

% Movie Recommender & get\_movies
% & 1 \\

% Wikipedia & --- & 0 \\

% Reddit & --- & 0 \\

% \bottomrule
% \end{tabular}
% \label{tab:mcp-bench-tools}
% \caption{Summary of MCP Servers and Tool Counts}
% \end{table}
% \clearpage




% \subsection{Unified Agent}

% The unified agent unifies all the agents together under a single framework, meaning a single agent with access to tool calls for all the different tasks. We will try to combine the system prompts of all the different agents together into a single system prompt that will work for all tasks. We expect performance to decrease, but maybe we can derive insights from this practice.\\
% \andy{
% Implementation will be quite tough, as the different agents call tools in different ways, TBD how to approach this.
% }

% \section{Analysis}

% \subsection{Test-Time Scaling}

% For test-time scaling, we will analyze:
% \begin{itemize}
% \item \textbf{Context Length:} This is the total length of a trajectory of an agent, including the following components:
% \begin{itemize}
% \item System prompt
% \item Question Input
% \item Model Output and Tool-Calls over multiple iterations (not including reasoning)
% \item Environment Output
% \end{itemize}
% \item \textbf{Avg. Reasoning Tokens per Iteration}: This is the average amount of reasoning tokens for each iteration of agent:
% \begin{itemize}
% \item For close-sourced models, reasoning tokens is outputted by the API or LiteLLM
% \item For open-source models (besides R1), there is no notion of reasoning. We will use think tags to simulate reasoning within the output. These reasoning tokens will not be fed in to later iterations of the agent, to stay in line with closed-source models. However, this is a departure from the REACT agent framework.
% \end{itemize}
% \end{itemize}
% \andy{
% Search agent might need to be edited to not feed back in the reasoning for open-source models
% }

% \subsection{Lexical Analysis}

% Current work by Pranav keeps track of "reasoning words" within different categories:
% \begin{itemize}
% \item \textbf{Planning/Directive}: "let me", "i should", "i need to", "the goal is", "first", "next", "then", "the plan is", "now i will"
% \item \textbf{Hesitation/Correction}: "wait", "maybe", "alternatively", "hmm", "perhaps"
% \item \textbf{Cognitive/Assertive}: "i think", "it seems", "this means", "therefore", "however", "based on", "okay", "good"
% \end{itemize}
% Then we plot the following for each model:
% \begin{itemize}
% \item avg. reasoning words per trajectory
% \item avg. count of each category of reasoning words per trajectory
% \item avg. reasoning words vs. accuracy
% \end{itemize}
% \andy{
% This metric doesn't seem like it'll work for the closed-source models, since we don't have access to the reasoning trace?
% }
% \xiaochuan{Yes. We can only do this analysis on open-source models.}

% \subsection{Reflection Ability (to be finalized)}

% We observe that models in agentic tasks tend to spend substantial time planning initially and rarely revise their plans later—showing limited reflection or self-correction. The models often appear highly confident in their initial answers.

% This may relate to entropy analysis: early tokens exhibit higher entropy (representing planning), while later tokens show lower entropy (execution).
% A possible experiment is to manually add an instruction such as “Double-check your answer and adjust your plan,” then measure the proportion of cases where the model changes its response:
% \begin{itemize}
%     \item From wrong → correct
%     \item From correct → wrong
%     \item From wrong → wrong (ineffective reflection)
% \end{itemize}

% \subsection{Sparse Attention}









\section{Detailed Prompts}\label{appendix:prompt}

\paragraph{Universal Agent Prompt}
The agent system prompt instructs the model to solve diverse problems using tools or reasoning. It emphasizes careful tool selection, avoiding redundant calls, and building on previous results. For Tau2-Bench tasks, domain-specific policy documents are appended under a \textit{Policy} section, loaded from policy markdown file. Other benchmarks  use the base prompt without policy appended.

\begin{tcolorbox}[colback=gray!5,colframe=gray!75,title=Universal Agent System Prompt,fonttitle=\bfseries\small,fontupper=\ttfamily\scriptsize,breakable]
You are a helpful AI agent that can solve a wide variety of problems, including searching the web for information, writing and running code, performing calculations and logical reasoning, and interacting with external services. You can choose to use tools when helpful, or solve problems through your own reasoning.\\[0.5em]
\#\# Tool Selection\\
- CAREFULLY read tool names and descriptions before selecting\\
- Choose tools that are DIRECTLY relevant to the current task\\
- AVOID REDUNDANT CALLS: Don't repeat successful tools unless specifically needed\\
- If no tools are needed, solve the problem through reasoning alone\\[0.5em]
\#\# Execution Strategy\\
- Analyze the task to understand what information or actions are needed\\
- Decide whether to use tools, reason independently, or combine both approaches\\
- If using tools, identify which are most relevant based on their names and descriptions\\
- BUILD ON PREVIOUS RESULTS: Use information from previous tool calls\\
- If a tool returns an error, try an alternative approach or tool\\[0.5em]
\#\# Response Guidelines\\
- Follow any policies (<policy>) or constraints provided in the task\\
- When you have gathered sufficient information, provide a clear final answer\\
- If a task cannot be completed with available tools, try solving it through reasoning\\
- Do not guess or make up information - only use data from tool results or verified reasoning
\end{tcolorbox}

\paragraph{Point-wise Self-choice Prompt}
This prompt is used in point-wise self-choice. Our prompt first positions the model as a ``Lead Researcher'' auditing agent behavior. Our prompt prioritizes result correctness over reasoning quality---even flawed processes leading to correct answers receive ``Correct'' judgments. The model outputs structured $<\text{judgment}>\text{Correct}/\text{Wrong}</\text{judgment}>$ tags, extracted via regex for downstream evaluation against ground truth scores.

\begin{tcolorbox}[colback=gray!5,colframe=gray!75,title=Point-wise Self-choice System Prompt,fonttitle=\bfseries\small,fontupper=\ttfamily\scriptsize,breakable]
You are a Lead Researcher specializing in auditing autonomous AI agents. You are rigorous, detail-oriented, and objective. You will be provided with a trajectory (a history of an agent's thoughts, tool calls, and environment observations) and the original Task Description. Evaluate whether the answer is correct.\\[0.5em]
First, look strictly at the final answer or final state. Does it satisfy the user's request?\\
Second, review the steps. Did the agent make logical errors, hallucinate, or misuse tools?\\[0.5em]
Your primary job is to determine if the Final Answer is correct. Even if the reasoning was flawed (e.g., accidental discovery, wrong logic) but the Final Answer is factually correct, you MUST mark the answer as ``Correct''. You should note the process error separately. You must cite specific evidences or observations to support your critique. Meanwhile, don't trust the answers too much.\\[0.5em]
After detailed thinking, return your final judgment within the tag <judgment>YOUR FINAL JUDGMENT</judgment>. You can only choose from Correct or Wrong. \\

\#\# Available Tools: The agent had access to the following tools: \{\{ Standard Tool Schema\}\} \\

-- \\

\#\# Task Description:  \{\{Task Description\}\}

--- \\

\#\# Trajectory: \{\{Trajectory\}\}

---\\

Please evaluate whether the agent's final answer correctly addresses the task. Provide your judgment in <judgment>YOUR FINAL JUDGMENT</judgment>. You can only choose from Correct or Wrong.

\end{tcolorbox}

\paragraph{Pair-wise Self-choice Prompt}
This prompt is used in pair-wise self-choice. Our prompt compares two trajectories to identify the superior answer. The model evaluates final answers and reasoning quality for both, outputting $<\text{ranking}>\text{1}/\text{2}</\text{ranking}>$ to indicate preference. We adopt the bump-sort algorithm in pair-wise self-choice, which performs $n-1$ pairwise comparisons across $n$ passes and claims the best response across all responses.

\begin{tcolorbox}[colback=gray!5,colframe=gray!75,title=Pair-wise Self-choice System Prompt,fonttitle=\bfseries\small,fontupper=\ttfamily\scriptsize,breakable]
You are a Lead Researcher specializing in auditing autonomous AI agents. You are rigorous, detail-oriented, and objective. You will be provided with the original Task Description and TWO trajectories (each consisting of a history of an agent's thoughts, tool calls, and environment observations). Evaluate which trajectory produced the better answer.\\[0.5em]
First, look strictly at the final answer or final state of each trajectory. Does it satisfy the user's request?\\
Second, review the steps of each trajectory. Did the agent make logical errors, hallucinate, or misuse tools?\\[0.5em]
Your primary job is to determine which Final Answer is better. Even if the reasoning was flawed (e.g., accidental discovery, wrong logic) but the Final Answer is factually superior, you MUST mark that trajectory as the better one. You should note the process errors separately. You must cite specific evidences or observations to support your critique.\\[0.5em]
After detailed thinking, return your final preference within the tag <ranking>YOUR FINAL PREFERENCE</ranking>. You can only choose 1 or 2.

\#\# Available Tools: The agent had access to the following tools: \{\{ Standard Tool Schema\}\} \\

-- \\

\#\# Task Description:  \{\{Task Description\}\}

--- \\

\#\# Trajectory 1: \{\{Trajectory 1\}\}

---\\

\#\# Trajectory 2: \{\{Trajectory 2\}\}

---\\


Please evaluate which trajectory produced the better final answer. Provide your analysis and return your preference (1 or 2) in <ranking>YOUR FINAL PREFERENCE</ranking>.

\end{tcolorbox}

















\end{document}

% This document was modified from the file originally made available by
% Pat Langley and Andrea Danyluk for ICML-2K. This version was created
% by Iain Murray in 2018, and modified by Alexandre Bouchard in
% 2019 and 2021 and by Csaba Szepesvari, Gang Niu and Sivan Sabato in 2022.
% Modified again in 2023 and 2024 by Sivan Sabato and Jonathan Scarlett.
% Previous contributors include Dan Roy, Lise Getoor and Tobias
% Scheffer, which was slightly modified from the 2010 version by
% Thorsten Joachims & Johannes Fuernkranz, slightly modified from the
% 2009 version by Kiri Wagstaff and Sam Roweis's 2008 version, which is
% slightly modified from Prasad Tadepalli's 2007 version which is a
% lightly changed version of the previous year's version by Andrew
% Moore, which was in turn edited from those of Kristian Kersting and
% Codrina Lauth. Alex Smola contributed to the algorithmic style files.
